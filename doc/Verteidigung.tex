\documentclass[10pt]{beamer}
\usepackage[utf8x]{inputenc}
\usepackage[ngerman]{babel}
\usepackage{amsmath}

\usepackage{verbatim}

\usepackage{multicol} %Spalten

\setlength{\parskip}{0.5em} % Abstand nach Absatz

\usetheme{Dresden} %http://www.hartwork.org/beamer-theme-matrix/
%\usetheme{Goettingen} 
\usecolortheme{seahorse} %http://www.hartwork.org/beamer-theme-matrix/
\setbeamercovered{transparent}

%%%
% Colors
%
\definecolor{orange}{rgb}{.99,.78,0}
\definecolor{blau}{rgb}{0,.52,.73}
\definecolor{blaurot}{rgb}{.8,.52,.73}
\newcommand{\liquid}[1]{\textcolor{blau}{#1}}
\newcommand{\liquidcenter}[1]{\textcolor{blaurot}{#1}}
\newcommand{\wichtig}[1]{\textit{#1}}
\newcommand{\gas}[1]{\textcolor{orange}{#1}}

%%%
% Quellcode
%
\usepackage{xcolor}
\usepackage{listings}

\definecolor{dkgreen}{rgb}{0,0.6,0}
\definecolor{mygray}{rgb}{0.5,0.5,0.5}
\definecolor{mymauve}{rgb}{0.58,0,0.82}
\lstset{ %
	backgroundcolor=\color{white},   % choose the background color; you must add \usepackage{color} or \usepackage{xcolor}
	basicstyle=\scriptsize,        % the size of the fonts that are used for the code
	breakatwhitespace=false,         % sets if automatic breaks should only happen at whitespace
	breaklines=false,                 % sets automatic line breaking
	captionpos=b,                    % sets the caption-position to bottom
	commentstyle=\color{dkgreen},    % comment style
	escapeinside={\%*}{*)},          % if you want to add LaTeX within your code
%	frame=single,                    % adds a frame around the code
	keepspaces=true,                 % keeps spaces in text, useful for keeping indentation of code (possibly needs columns=flexible)
	keywordstyle=\color{blue},       % keyword style
	numbers=left,                    % where to put the line-numbers; possible values are (none, left, right)
	numbersep=5pt,                   % how far the line-numbers are from the code
	numberstyle=\tiny\color{mygray}, % the style that is used for the line-numbers
	rulecolor=\color{black},         % if not set, the frame-color may be changed on line-breaks within not-black text (e.g. comments (green here))
	showspaces=false,                % show spaces everywhere adding particular underscores; it overrides 'showstringspaces'
	showstringspaces=false,          % underline spaces within strings only
	showtabs=false,                  % show tabs within strings adding particular underscores
	stepnumber=2,                    % the step between two line-numbers. If it's 1, each line will be numbered
	stringstyle=\color{mymauve},     % string literal style
	tabsize=2,                       % sets default tabsize to 2 spaces
	title=\lstname                   % show the filename of files included with \lstinputlisting; also try caption instead of title
}

%%%
% Wide frames (low margin)
%
\usepackage{calc}
\usepackage{environ}

\newcommand{\halfmargin}{0.05\paperwidth}
\newcommand{\margin}{0.10\paperwidth}

%\beamersetrightmargin{\margin}
%\beamersetleftmargin{\margin}

\NewEnviron{wideframe}[1][]{%
	\begin{frame}{#1}
		\makebox[\textwidth][c]{
			\begin{minipage}{\dimexpr\paperwidth-\halfmargin}
				\BODY
			\end{minipage}}
		\end{frame}
	}


%%%
% Page number
%
\addtobeamertemplate{navigation symbols}{}{%
	\usebeamerfont{footline}%
	\usebeamercolor[fg]{footline}%
	\hspace{1em}%
	\insertframenumber/\inserttotalframenumber
}

%%%
% Title
%
\title[Visualisierung von Strukturveränderungen in Molekulardynamikdaten]{Visualisierung von Strukturveränderungen in Molekulardynamikdaten}
\subtitle[Bachelorarbeit Verteidigung]{Bachelorarbeit Verteidigung}
\author[Richard Hähne]{Richard Hähne}
\institute[Fakultät Informatik – Institut SMT - Professur für Computergraphik und Visualisierung]{Institut für Software- und Multimediatechnik\\Professur für Computergraphik und Visualisierung}
\date{\today}
 
\begin{document}
\maketitle
%\frame{\tableofcontents[hideallsubsections]}

\section{Einleitung}
%\frame{\tableofcontents[currentsection, hideothersubsections]}

\subsection{Strukturereignisse während einer Molekulardynamiksimulation}

%
% Diffusion in dünnen Schichten bei Temperatureinfluss (Eigenschaftsveränderung der Oberfläche (Farbe, Härte etc), Haftung)
% mikroskopische Abtragung der Beschichtung von Turbinenblättern durch lokale Druckveränderungen
%
\begin{frame}[<+->]{Motivation}
	\begin{itemize}
		\item Molekulardynamik betrachtet Atome und Moleküle (\wichtig{Partikel}) begrenzter Zahl über kurze Zeiträume
		\item durch räumliche Nähe der \wichtig{Partikel} bilden sich Strukturen
		\item hier: Phasenübergang zwischen Gas und Flüssigkeit begrenzt Strukturen
		\item Partikel bewegen sich, Strukturen werden verändert
		\item über die Struktur-Eigenschafts-Beziehungen der \wichtig{Partikel} können Vorhersagen über Materialverhalten getroffen werden
		\item daher ist es sinnvoll, Strukturveränderungen zu erkennen
	\end{itemize}
\end{frame}

%	- zwei Phasen: Flüssig (blaurot)- und Gasphase (gelb)
\begin{frame}{Vorliegender Datensatz}
	\begin{columns}[T]
		\begin{column}{0.5\textwidth}
			%TODO: Bild Distanz!
			\includegraphics*<5->[width=\textwidth]{mediaDefense/SignedDistanceColor-Show-Frame-60}
		\end{column}
		\begin{column}{0.5\textwidth}
			\begin{itemize}[<+->]
				\item zerreißender Flüssigkeitsfilm
				\item 150 Zeitschritte
				\item 2 Millionen Partikel
				\item Partikelposition je Zeitschritt
				\item Partikelabstand zur am nächsten gelegenen Phasengrenze
			\end{itemize}
		\end{column}
	\end{columns}
\end{frame}

%
%	Animation:
%	- Flüssigkeitsfilm umgeben von Vakuum -> expandiert rapide
%	Ziel:
%	- es soll erkannt werden, was währenddessen mit der Struktur passiert -> Strukturereignisse
%
\begin{frame}
	\begin{columns}[T]
		\begin{column}{5.5cm}
			Expandierende Flüssigkeit:\\
			\visible<1->{
				{\scriptsize $t_S=0$}\\
				\includegraphics*[width=5.5cm]{mediaDefense/SignedDistanceColor-Frame-000-letterbox.png}
			}\\

			\visible<2->{
				{\scriptsize $t_S=20$}\\		
				\includegraphics*[width=5.5cm]{mediaDefense/SignedDistanceColor-Frame-020-letterbox.png}
			}\\

			\visible<3->{
				{\scriptsize $t_S=30$}\\		
				\includegraphics*[width=5.5cm]{mediaDefense/SignedDistanceColor-Frame-030-letterbox.png}
			}\\
			
			\visible<4->{
				{\scriptsize $t_S=100$}\\		
				\includegraphics*[width=5.5cm]{mediaDefense/SignedDistanceColor-Frame-100-letterbox.png}
			}\\
		\end{column}
		\begin{column}{6.5cm}
			\visible<5->{
				Auftretende \wichtig{Strukturereignisse}:\\
				\includegraphics*[width=6.5cm]{mediaDefense/Strukturereignisse.png}
			}
		\end{column}
	\end{columns}
\end{frame}

%TODO evtl Folie mit Video

\subsection{Erkennung und Visualisierung der Strukturereignisse}

\begin{frame}{Ziel}
	Wie verhält sich die Flüssigkeit, welche Strukturveränderungen treten auf?
	
	Die vorliegende Animation ist zur Analyse nur bedingt geeignet:
	\begin{itemize}
		\item vieles geschieht gleichzeitig: Erfassung schwierig
		\item sequentieller Ablauf: quantitativer Vergleich schwierig
		%\item Diagramm: unübersichtlich bei vielen und übereinanderlagernden Daten
	\end{itemize}
	Lösung: Darstellung der zeitlichen Entwicklung der Strukturveränderungen im geometrischen Kontext der Originaldaten.
\end{frame}

\begin{frame}{Ziel}
	TODO: Makrobild, Seitenansicht, mit überlagerter Darstellung
\end{frame}

% 1) Analyse der Struktur: Analyse der Partikelstruktur für jeden Zeitschritt
% 2) Erkennung von Veränderungen Erkennung von Änderungen an der Struktur von einem zum nächsten Zeitschritt (Strukturereignisse)
% 3) Darstellung: Anzeige dieser Ereignisse im Ortsraum der Partikel
\begin{wideframe}
	\includegraphics*[width=\textwidth]{mediaDefense/VisMD-Ablaufschema-EinAusgang}
\end{wideframe}

\begin{frame}{Herausforderungen}
	\begin{itemize}
		\item Sinnvolle Einteilung der Partikel in Strukturen, um Ereignisse erkennen zu können.
		\item Vorgehen, um Veränderungen der Struktur über die Zeit verfolgen zu können.
		\item Berechnungsdauer überschaubar halten (Überprüfung der Algorithmen).
	\end{itemize}
\end{frame}

%
% Skelettext: Oberflächen, Volumen, Punktdaten: umfangreiche Pre- oder Postprocessingverfahren zum Entfernen von Artefakten
% Konturbäume (u.a. Bajaj et al., Carr et al., temporal: u.a. Laney et al., Bajaj et al.): Datenelemente anhand ihres Funktionswertes zu gruppieren und für verschiedene Funktionswerte Zusammenhangskomponenten erkennen zu können
% basierend auf Skelettextraktion (Nachbarschaftsbildung über Reichweite: Klasing et al.)
% basierend auf Konturbaum: Nutzung kritischer Funktionswert im CFD: Carr et al.
%
\begin{frame}{Verwandte Arbeiten}
	\begin{itemize}
		\item Strukturanalyse großer Datensätze
		\begin{itemize}
			\item Skelettextraktion
			\item Konturbäume (Funktionswert, Zusammenhangskomponenten)
			\item Nutzung von Konturbäumen zur 
		\end{itemize}
		\item Element- und Zeitdarstellung in Molekulardynamiksimulationen
		\begin{itemize}
			\item Elemente sind meist dreidimensional, farbkodiert
			\item Zeitdarstellung durch Animation oder Diagramme
		\end{itemize}
		\item Visualisierung
		\begin{itemize}
			\item Anordnung und Filterung von Datenobjekten
			%\item Anordnung (Gleicher et al.) und Filterung von Datenobjekten (Lima)
			\item Glyphdesign
		\end{itemize}
	\end{itemize}
\end{frame}

\begin{frame}{Einordnung dieser Arbeit}
	\begin{itemize}
		\item Strukturuntersuchung von Punktdaten
		\item Visualisierung (zeitabhängiger) Elemente im Ortsraum visualisierter Punktdaten
%		\item einfacher Algorithmus für die Strukturanalyse, basieren auf Ideen der Skelettextraktion und Konturbäume
%		\item Verfolgung zeitveränderlicher Daten ohne Baumstrukturen
	\end{itemize}
\end{frame}


\section{Analyse der Partikelstruktur}
%\frame{\tableofcontents[currentsection, hideothersubsections]}

%	- Begriff CLUSTER muss hier fallen! -> Einteilung der Partikel in Cluster
\begin{wideframe}
	\includegraphics*[width=\textwidth]{mediaDefense/VisMD-Ablaufschema-Strukturanalyse}
\end{wideframe}

\subsection{Anforderungen}

\begin{frame}Cluster als Zusammenhangskomponenten
	\includegraphics*[width=11cm]{mediaDefense/cluster/SignedDistanceColor-Frame-090-clipplane20-letterbox.png}\\
	\includegraphics*[width=11cm]{mediaDefense/cluster/SignedDistanceColor-Frame-090-clipplane20-letterbox-Cluster.png}
%	Einteilung der Partikelmenge in Cluster!
\end{frame}

\begin{frame}Cluster innerhalb von Zusammenhangskomponenten
	\includegraphics*[width=11cm]{mediaDefense/cluster/SignedDistanceColor-Frame-016-clipplane20-letterbox.png}\\
	\includegraphics*[width=11cm]{mediaDefense/cluster/SignedDistanceColor-Frame-016-clipplane20-letterbox-Cluster.png}
\end{frame}

\subsection{Cluster Fast Depth (CFD) Algorithmus}
% Inhalt dieses Frames während der Bilder erklären
%\begin{frame}
%	\begin{itemize}
%		\item alle Partikel werden sequentiell durchlaufen
%		\item dabei werden \gas{Gaspartikel} ignoriert und nur Partikel der \liquid{Flüssigkeitsphase} betrachtet, die noch keinem Cluster zugeordnet sind
%		\item ausgehend von einem solchen Partikel werden seine Nachbarn auf ihren \wichtig{Tiefenwert} hin untersucht
%		\item der Nachbar mit dem höchsten Wert wird selektiert und es wird wiederrum der \wichtig{Distanzwert} von dessen Nachbarn betrachtet
%		\item diese Schleife läuft so lang, bis alle Nachbarn denselben oder einen geringeren \wichtig{Distanzwert} aufweisen (\liquidcenter{lokales Maximum} erreicht)
%		\item alle auf dem Weg selektierten Partikel werden anschließend einem (neuen oder vorhandenen) Cluster zugeordnet
%	\end{itemize}
%\end{frame}

%
% Zusammenhangskomponente vgl "Cluster innerhalb von Zusammenhangskomponenten"
% Gas wäre außen drumherum: Interessiert nicht.
% Große Zahl = Distanz, kleine = ID der Partikel
% 0 = Grenze des Phasenübergangs: geschlossen, über nutzerdefinierten Wertebereich für die lokale Dichte bestimmt
% idealisiertes Beispiel (Distanz Gleitkommawert)
%
\begin{frame}{Zusammenhangskomponente}
	\includegraphics*[width=11cm]{mediaDefense/cluster/cfd.png} \\
\end{frame}

\begin{frame}
	\includegraphics*[width=3.7cm]{mediaDefense/cluster/cfd000.png} \\
	\includegraphics*[width=3.7cm]{mediaDefense/cluster/cfd001.png} \\
	\includegraphics*[width=3.7cm]{mediaDefense/cluster/cfd002.png} \\
	\includegraphics*[width=3.7cm]{mediaDefense/cluster/cfd003.png}
\end{frame}

\begin{frame}
	\includegraphics*[width=3.7cm]{mediaDefense/cluster/cfd000.png} \includegraphics*[width=3.7cm]{mediaDefense/cluster/cfd004.png} \\
	\includegraphics*[width=3.7cm]{mediaDefense/cluster/cfd001.png} \includegraphics*[width=3.7cm]{mediaDefense/cluster/cfd005.png} \\
	\includegraphics*[width=3.7cm]{mediaDefense/cluster/cfd002.png} \includegraphics*[width=3.7cm]{mediaDefense/cluster/cfd006.png} \\
	\includegraphics*[width=3.7cm]{mediaDefense/cluster/cfd003.png} \includegraphics*[width=3.7cm]{mediaDefense/cluster/cfd007.png}
\end{frame}

\begin{frame}
	\includegraphics*[width=3.7cm]{mediaDefense/cluster/cfd000.png} \includegraphics*[width=3.7cm]{mediaDefense/cluster/cfd004.png} 	\includegraphics*[width=3.7cm]{mediaDefense/cluster/cfd008.png} \\
	\includegraphics*[width=3.7cm]{mediaDefense/cluster/cfd001.png} \includegraphics*[width=3.7cm]{mediaDefense/cluster/cfd005.png} 	\includegraphics*[width=3.7cm]{mediaDefense/cluster/cfd009.png} \\
	\includegraphics*[width=3.7cm]{mediaDefense/cluster/cfd002.png} \includegraphics*[width=3.7cm]{mediaDefense/cluster/cfd006.png} 	\includegraphics*[width=3.7cm]{mediaDefense/cluster/cfd010.png} \\
	\includegraphics*[width=3.7cm]{mediaDefense/cluster/cfd003.png} \includegraphics*[width=3.7cm]{mediaDefense/cluster/cfd007.png}	\includegraphics*[width=3.7cm]{mediaDefense/cluster/cfd011.png}
\end{frame}

%
% Schön zu sehen sind die gleiche Färbung nebeneinanderliegender Cluster
%
\begin{frame}{Überprüfung der erzeugten Cluster}
	Einfärbung der Partikel zur Clusterüberprüfung:
	\begin{itemize}
		\item zufällige Färbung in jedem Zeitschritt (links)
		\item Färbung anhand von Clusterparametern und
		\item Vererbung vom vorhergehenden Zeitschritt (rechts)
	\end{itemize}
	
	\includegraphics*[width=5.7cm]{mediaDefense/eva/clusterfarbe-allRd-small.png}
	\includegraphics*[width=5.7cm]{mediaDefense/eva/clusterfarbe-rdInteritance-small.png}
	
	Vererbung geschieht unter Nutzung des Clustervergleichs. % über den ich jetzt rede
\end{frame}

\section{Erkennung der Ereignisse}
%\frame{\tableofcontents[currentsection, hideothersubsections]}

%
% SECC: Vergleich von Clustern zweier aufeinanderfolgender Zeitschritte
% Heuristik zur Ableitung von Ereignissen
%
\begin{wideframe}
	\includegraphics*[width=\textwidth]{mediaDefense/VisMD-Ablaufschema-Ereigniserkennung}
\end{wideframe}

\subsection{Clustervergleich (SECC)}

%
% zwei aufeinanderfolgende Zeitschritte
% Zusammenhang durch gemeinsame Partikel
%
\begin{frame}{Clustervergleichsmatrix}

	TODO: Bild Clustervergleichsmatrix vorauss. 5x5 nötig:\\
	mit ClusterIDs und Kennzeichnung aktueller/vorhergehender Zeitschritt,\\
	mit Clustergröße,\\
	mit Partikelmengen

	%\item diese Zuordnung wird für zwei aufeinanderfolgende Zeitschritte betrachtet (aktueller und vorheriger Zeitschritt)
	%\item die Anzahl der Partikel pro Cluster wird in einer n-m-Matrix festgehalten, wobei n für die Anzahl an Clustern des aktuellen und m für die Clusteranzahl des vorhergehenden Zeitschritts steht
	%\item der Zeilenindex steht für die Cluster ID des aktuellen und der Spaltenindex für die Cluster ID des vorhergehenden Zeitschritts
	%\item die Zellen enthalten die zugehörige Partikelanzahl; semantisch ist dies die Anzahl gemeinsamer Partikel der Cluster zweier Zeitschritte
\end{frame}

\begin{frame}{Partnercluster}
	
	TODO: Bild Clustervergleichsmatrix:\\
	Spalten und Zeileneinfärbung (Rückwärts-/Vorwärtsgerichteter Vergleich)
	
	%\item die \wichtig{Clustervergleichsmatrix} wird sowohl Zeilen- als auch Spaltenweise betrachtet
	%\item dadurch kann das Verhältnis hinsichtlich der Partikelanzahl eines Clusters des aktuellen Zeitschrittes zu allen Clustern des vorhergehenden Zeitschrittes ermittelt werden (und umgekehrt)
	%\item diese zwei Richtungen werden vorwärtsgerichteter Vergleich (alter Zeitpunkt zu aktuellem) und rückwärtsgerichteter Vergleich (aktueller zu altem Zeitpunkt) genannt
	%\item Cluster mit gemeinsamen Partikeln werden als Partner bezeichnet
\end{frame}

\begin{frame}{Relevante Werte}
	TODO: Bild Clustervergleichsmatrix:\\
	Summen und Anteile in Extrazeile/Spalte außen dran\\
	Große Partner eventuell erst bei Split/Merge hervorheben (dort erst gebraucht!)
	
	Kann u.U. mit vorhergehender Folie verschmelzen (hängt von der Komplexität der Grafik ab).

	%\item Gesamtanzahl der Partner
	%\item Anzahl an großen Partnern. Große Partner sind solche Partnercluster mit vielen gemeinsamen Partikeln.
	%\item Grenzwert, ab wann ein Partner ein großer Partner ist. Der Grenzwert ist das Verhältnis der gemeinsamen Partikel des Partners im Vergleich zu der Gesamtzahl gemeinsamer Partikel des Clusters.
\end{frame}

\subsection{Ereignisheuristik}

\begin{frame}{Birth und Death}
	
	TODO: Bild Clustervergleichsmatrix:\\
	Spalte ohne gemeinsame Anteile\\
	Zeile ohne gemeinsame Anteile
	
	%\item Gesamtanzahl der Partner ist Null
	%\item Birth im rückwärtsgerichteten Vergleich: kein Partner des vorangegangenen Zeitschrittes enthält Partikel des derzeitigen Clusters
	%\item Death im vorwärtsgerichteten Vergleich: kein Partner des aktuellen Zeitschrittes enthält Partikel des vorangegangenen Clusters
\end{frame}


\begin{frame}{Merge und Split}
	
	TODO: Bild Clustervergleichsmatrix:\\
	Spalte mit großen Partnern und Grenzwert\\
	Spalte mit großen Partnern und Grenzwert
	
	%\item Mindestanzahl großer Partner
	%\item Grenzwert für diese Partner (ein Grenzwert für alle)
	%\item Merge im rückwärtsgerichteten Vergleich: mehrere große Partner des vorangegangenen Zeitschrittes sind zum derzeitigen Cluster verschmolzen
	%\item Split im vorwärtsgerichteten Vergleich: mehrere große Partner des aktuellen Zeitschrittes sind aus einem vorangegangenen Cluster hervorgegangen
\end{frame}

\section{Visualisierung}
%\frame{\tableofcontents[currentsection, hideothersubsections]}
\begin{wideframe}
	\includegraphics*[width=\textwidth]{mediaDefense/VisMD-Ablaufschema-Visualisierung}
\end{wideframe}

\subsection{Strukturereignistaxonomie}

%
% Ereignisparameter
% Einteilung visueller Variablen
% numerische Werte (metrische Skala) und Kategorien (Nominalskala) mit vorstellen
%
\begin{frame}{Ereignisparameter und visuelle Variablen}
	\includegraphics*[width=\textwidth]{mediaDefense/vis/strukturereignistaxonomie}
\end{frame}

%
% Position im Raum (Aufgabenstellung)!
% Zeitpunkt: Farbwert, Helligkeit, Füllstand (Textur)
% Art: abstrakt und narrativ symbolische Darstellung (Textur)
%
\begin{frame}{Zuweisung der visuellen Variablen}
	\includegraphics*[width=\textwidth]{mediaDefense/vis/strukturereignistaxonomie-zuweisung}
\end{frame}

%
% Beispielhafte Zuweisung
% 3D-Darstellung schlecht:
% Die Erscheinung von Glyphen als Polygone ist blickwinkelabhängig (Form) und beleuchtungsabhängig (unterschiedliche Helligkeiten je nach Winkel der Flächen).
% Bewirkt Senkung der präattentiven Wirkung (kognitiver Aufwand). %präattentiv: Reiz erzeugen, der vom Nervensystem des Beobachters zwar wahrgenommen wird und einen Effekt erzeugt, jedoch nicht in das Bewusstsein dringt. Vorteil ist eine parallele, schnelle Verarbeitung (kürzer als 250	Millisekunden (ms)), die nicht gehemmt werden kann. Nachteilig ist, dass die Wahrnehmung	oberflächlich bleibt
% Erschwert Identifizierung bei Überlagerung mit der Partikeldarstellung (einzelne Polygone zwischen sehr vielen Kugeln) $\Rightarrow$ Wechsel zu 2D-Glyphen (Billboards).
%
\begin{frame}{Mockup mit beispielhafter Variablenzuweisung (WebGL)}
	\includegraphics*[height=6.9cm]{mediaDefense/vis/MockupZufallsdatenOben45bunt.png}
\end{frame}

\subsection{Glyphdarstellung} %	Glyph ist die grafische Darstellung eines Schriftzeichens!

\begin{frame}{Glyphen in situ}
	TODO: Abbildung Glyphen im Partikelraum (identisch zur Eingangsfolie <-> Wiedererkennung!)
\end{frame}

%
% Förderung präattentive Wahrnehmung: möglichst einfach gestaltet, wenig Variablen
% -> unterstützt durch Gesetz der Geschlossenheit
%
\begin{frame}{Glyphdarstellung Strukturereignis: Kodierung Ereignisart}
	Abstrakt symbolische Darstellung\\
					\includegraphics[width=.18\textwidth]{mediaDefense/vis/GlyphEventTypeAbstractBirthWhite-bg}
					\hspace*{.06\textwidth}%
					\includegraphics[width=.18\textwidth]{mediaDefense/vis/GlyphEventTypeAbstractDeathWhite-bg}
					\hspace*{.06\textwidth}%
					\includegraphics[width=.18\textwidth]{mediaDefense/vis/GlyphEventTypeAbstractMergeWhite-bg}
					\hspace*{.06\textwidth}%
					\includegraphics[width=.18\textwidth]{mediaDefense/vis/GlyphEventTypeAbstractSplitWhite-bg}
					
	Narrativ symbolische Darstellung\\
					\includegraphics[width=.18\textwidth]{mediaDefense/vis/GlyphEventTypeBirthWhite-bg}
					\hspace*{.06\textwidth}%
					\includegraphics[width=.18\textwidth]{mediaDefense/vis/GlyphEventTypeDeathWhite-bg}
					\hspace*{.06\textwidth}%
					\includegraphics[width=.18\textwidth]{mediaDefense/vis/GlyphEventTypeMergeWhite-bg}
					\hspace*{.06\textwidth}%
					\includegraphics[width=.18\textwidth]{mediaDefense/vis/GlyphEventTypeSplitWhite-bg}
					
	Der Rahmen erfüllt das Gesetz der Geschlossenheit. Seine Form hebt sich von der Kreisform ab (Kontrast zur Kugeldarstellung der Partikel) und gibt Hinweis auf exakte Ereignisposition.
\end{frame}

\begin{frame}{Glyphdarstellung Strukturereignis: Zeitdarstellung}
	%TODO Farbskala
	Füllstand (Textur)\\
					\includegraphics[width=.14\textwidth]{mediaDefense/vis/GlyphenEventTimeFill000}
					\hspace*{.1\textwidth}%
					\includegraphics[width=.14\textwidth]{mediaDefense/vis/GlyphenEventTimeFill050}
					\hspace*{.1\textwidth}%
					\includegraphics[width=.14\textwidth]{mediaDefense/vis/GlyphenEventTimeFill100}

	Helligkeit\\
					\includegraphics[width=.14\textwidth]{mediaDefense/vis/GlyphenEventTimeBrightness000}
					\hspace*{.1\textwidth}%
					\includegraphics[width=.14\textwidth]{mediaDefense/vis/GlyphenEventTimeBrightness050}
					\hspace*{.1\textwidth}%
					\includegraphics[width=.14\textwidth]{mediaDefense/vis/GlyphenEventTimeBrightness100}

	Farbwert\\
					\includegraphics[width=.14\textwidth]{mediaDefense/vis/GlyphenEventTimeHue000}
					\hspace*{.1\textwidth}%
					\includegraphics[width=.14\textwidth]{mediaDefense/vis/GlyphenEventTimeHue050}
					\hspace*{.1\textwidth}%
					\includegraphics[width=.14\textwidth]{mediaDefense/vis/GlyphenEventTimeHue100}
\end{frame}

\section{Ergebnisse}
\frame{\tableofcontents[currentsection, hideothersubsections]}

\subsection{Umfrage}
\begin{frame}
	\begin{itemize}
		\item Ziel: Bewertung der Visualisierung
		\item Zielgruppe: Laien und Professionelle
		\item Glyphdesign: intuitive Erkennung der Ereignisart, Zeitkodierung (Helligkeit: Erkennung ohne Farbskala) in Makrosicht, Positionierung, ...
		\item fachliche Interpretation Visualisierung
	\end{itemize}
\end{frame}

\begin{frame}
	Bilder der Fragestellung: Auswertungsbilder nur nennen
	oder maximal als Zahl

	\begin{itemize}
		\item 
	\end{itemize}
\end{frame}

\begin{wideframe}{Zusammenfassung}
	\includegraphics*[width=\textwidth]{mediaDefense/VisMD-Ablaufschema-komplett}
\end{wideframe}

\subsection{Bewertung}
\begin{frame}{Bewertung und Ausblick}

- ganz wenige Stichpunkte: was funktioniert gut (als Überleitung zur Zusammenfassung), was funktioniert nicht (als Überleitung zum Ausblick)

- Ausblick
	
	Erweiterung der Algos kann bei Ergebnisse kurz genannt werden
	Nutzung des Programms kann in Demo gezeigt werden
\end{frame}

\begin{frame}{Fragen}
	Fragen?
\end{frame}

\end{document}

