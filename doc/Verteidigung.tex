\documentclass[10pt]{beamer}
\usepackage[utf8x]{inputenc}
\usepackage[ngerman]{babel}
\usepackage{amsmath}

\usepackage{verbatim}

\usepackage{multicol} %Spalten

\setlength{\parskip}{0.5em} % Abstand nach Absatz

%%%
% Quellcode
%
\usepackage{xcolor}
\usepackage{listings}

\definecolor{dkgreen}{rgb}{0,0.6,0}
\definecolor{mygray}{rgb}{0.5,0.5,0.5}
\definecolor{mymauve}{rgb}{0.58,0,0.82}
\lstset{ %
	backgroundcolor=\color{white},   % choose the background color; you must add \usepackage{color} or \usepackage{xcolor}
	basicstyle=\scriptsize,        % the size of the fonts that are used for the code
	breakatwhitespace=false,         % sets if automatic breaks should only happen at whitespace
	breaklines=false,                 % sets automatic line breaking
	captionpos=b,                    % sets the caption-position to bottom
	commentstyle=\color{dkgreen},    % comment style
	escapeinside={\%*}{*)},          % if you want to add LaTeX within your code
%	frame=single,                    % adds a frame around the code
	keepspaces=true,                 % keeps spaces in text, useful for keeping indentation of code (possibly needs columns=flexible)
	keywordstyle=\color{blue},       % keyword style
	numbers=left,                    % where to put the line-numbers; possible values are (none, left, right)
	numbersep=5pt,                   % how far the line-numbers are from the code
	numberstyle=\tiny\color{mygray}, % the style that is used for the line-numbers
	rulecolor=\color{black},         % if not set, the frame-color may be changed on line-breaks within not-black text (e.g. comments (green here))
	showspaces=false,                % show spaces everywhere adding particular underscores; it overrides 'showstringspaces'
	showstringspaces=false,          % underline spaces within strings only
	showtabs=false,                  % show tabs within strings adding particular underscores
	stepnumber=2,                    % the step between two line-numbers. If it's 1, each line will be numbered
	stringstyle=\color{mymauve},     % string literal style
	tabsize=2,                       % sets default tabsize to 2 spaces
	title=\lstname                   % show the filename of files included with \lstinputlisting; also try caption instead of title
}

\usetheme{Dresden} %http://www.hartwork.org/beamer-theme-matrix/
%\usetheme{Goettingen} 
\usecolortheme{seahorse} %http://www.hartwork.org/beamer-theme-matrix/
\setbeamercovered{transparent}

\definecolor{orange}{rgb}{.99,.78,0}
\definecolor{blau}{rgb}{0,.52,.73}
\definecolor{blaurot}{rgb}{.8,.52,.73}
\newcommand{\liquid}[1]{\textcolor{blau}{#1}}
\newcommand{\liquidcenter}[1]{\textcolor{blaurot}{#1}}
\newcommand{\wichtig}[1]{\textit{#1}}
\newcommand{\gas}[1]{\textcolor{orange}{#1}}

% Page number
\addtobeamertemplate{navigation symbols}{}{%
	\usebeamerfont{footline}%
	\usebeamercolor[fg]{footline}%
	\hspace{1em}%
	\insertframenumber/\inserttotalframenumber
}

\title[Visualisierung von Strukturveränderungen in Molekulardynamikdaten]{Visualisierung von Strukturveränderungen in Molekulardynamikdaten}
\subtitle[Bachelorarbeit Verteidigung]{Bachelorarbeit Verteidigung}
\author[Richard Hähne]{Richard Hähne}
\institute[Professur für Computergraphik und Visualisierung]{Institut für Software- und Multimediatechnik\\Professur für Computergraphik und Visualisierung}
\date{\today}
 
\begin{document}
\maketitle
%\frame{\tableofcontents[hideallsubsections]}

\section{Einleitung}
%\frame{\tableofcontents[currentsection, hideothersubsections]}

\subsection{Strukturereignisse während einer Molekulardynamiksimulation}

\begin{frame}
	Partikelbasierte, zeitveränderliche Daten\\
	Erkennung von Änderungen der Struktur über die Zeit
	\includegraphics*[width=10.5cm]{media/SignedDistanceColor-Show-Frame-60.png}
%	Text:
%	- die Partikel sind eine Menge von Elementen mit diskreten Positionen
%	- zwei Phasen: Flüssig (blaurot)- und Gasphase (gelb)
\end{frame}

\begin{frame}
	\begin{columns}[T]
		\begin{column}{5.5cm}
		Expandierende Flüssigkeit:\\
		{\scriptsize $t_S=0$}\\
		\includegraphics*[width=5.5cm]{media/SignedDistanceColor-Frame-000-letterbox.png}\\
		%\includegraphics*[width=10cm]{media/SignedDistanceColor-Frame-010-letterbox.png}\\

		{\scriptsize $t_S=20$}\\		
		\includegraphics*[width=5.5cm]{media/SignedDistanceColor-Frame-020-letterbox.png}\\

		{\scriptsize $t_S=30$}\\		
		\includegraphics*[width=5.5cm]{media/SignedDistanceColor-Frame-030-letterbox.png}\\

		{\scriptsize $t_S=100$}\\		
		\includegraphics*[width=5.5cm]{media/SignedDistanceColor-Frame-100-letterbox.png}\\
		\end{column}
		\begin{column}{6.5cm}
			Auftretende \wichtig{Strukturereignisse}:\\
			TODO: Pfeil nach rechts hinzu
			\includegraphics*[width=6.5cm]{media/Strukturereignisse.png}
		\end{column}
	\end{columns}

%	Beschreibung Animation:
%	- Flüssigkeitsfilm umgeben von Vakuum -> expandiert rapide
%	Ziel:
%	- es soll erkannt werden, was währenddessen mit der Struktur passiert -> Strukturereignisse
\end{frame}

\subsection{Erkennung und Visualisierung der Strukturereignisse}
\begin{frame}
	Wie verhält sich die Flüssigkeit, welche Strukturereignisse treten auf?
	
	Die vorliegende Animation ist zur Analyse nur bedingt geeignet:
	\begin{itemize}
		\item vieles geschieht gleichzeitig: Erfassung schwierig
		\item sequentieller Ablauf: quantitiativer Vergleich schwierig
		%\item Diagramm: unübersichtlich bei vielen und übereinanderlagernden Daten
	\end{itemize}
	Lösung: Darstellung der zeitlichen Entwicklung der Struktur der Partikel im geometrischen Kontext der Originaldaten. Zwei Schritte erforderlich:
	\begin{enumerate}
		\item Analyse der Struktur
		\item Darstellung der Analyseergebnisse
	\end{enumerate}
	
%	mit dieser Arbeit wurden beide Schritte durchgeführt
\end{frame}


\begin{frame}
	\includegraphics*[width=11.5cm]{media/VisMD-Ablaufschema}
	
	TODO: Schema ohne Pfeilbeschriftungen/ohne UserInput (baut sich über den Vortragsverlauf in drei Schritten auf), Nachbar/CFD hervorheben

	%	- Analyse der Partikelstruktur für jeden Zeitschritt
	%	- Erkennung von Änderungen an der Struktur von einem zum nächsten Zeitschritt (Strukturereignisse)
	%	- Anzeige dieser Ereignisse im Ortsraum der Partikel
	%
	%	- Begriff CLUSTER muss hier fallen! -> Einteilung der Partikel in Cluster
\end{frame}

\section{Analyse der Partikelstruktur}
%\frame{\tableofcontents[currentsection, hideothersubsections]}

\subsection{Anforderungen}

\begin{frame}Cluster als Zusammenhangskomponenten
	\includegraphics*[width=11cm]{media/cluster/SignedDistanceColor-Frame-090-clipplane20-letterbox.png}\\
	\includegraphics*[width=11cm]{media/cluster/SignedDistanceColor-Frame-090-clipplane20-letterbox-Cluster.png}
%	Einteilung der Partikelmenge in Cluster!
\end{frame}

\begin{frame}Cluster innerhalb von Zusammenhangskomponenten
	\includegraphics*[width=11cm]{media/cluster/SignedDistanceColor-Frame-016-clipplane20-letterbox.png}\\
	\includegraphics*[width=11cm]{media/cluster/SignedDistanceColor-Frame-016-clipplane20-letterbox-Cluster.png}
\end{frame}

\begin{frame}{Einteilung der Partikelmenge in eine Menge von Clustern}

	Zur Verfügung stehende Daten für die Clustereinteilung:
	\begin{itemize}
		\item Partikelposition (durch Datensatz gegeben)
		\item Partikeldistanz zur nächsten Grenzfläche (durch Datensatz gegeben):
		\begin{itemize}
			\item die Grenzfläche zwischen Flüssig- und Gasphase ist bekannt
			%basiert auf einem Grenzwert in der lokalen Dichte basiert
			\item gegeben für jeden Partikel ist der Abstand zur nächsten Grenzfläche (\wichtig{Distanz})
			%(vorzeichenbehaftete Distanzfunktion):
			%Partikel mit \gas{negativem} Abstand liegen in der Gasphase
			%Partikel mit \liquid{positivem} Abstand liegen in der flüssigen Phase 
		\end{itemize}
		\item Nachbarschaftsbeziehungen der Partikel (mit Hilfe eines kD-Baums berechnet)
	\end{itemize}
	TODO: Evtl Bild für Distanz einfügen, allerdings wird dies durch das folgende CFD Schema sichtbar
\end{frame}
	
\subsection{Cluster Fast Depth (CFD) Algorithmus}
% Inhalt dieses Frames während der Bilder erklären
%\begin{frame}
%	\begin{itemize}
%		\item alle Partikel werden sequentiell durchlaufen
%		\item dabei werden \gas{Gaspartikel} ignoriert und nur Partikel der \liquid{Flüssigkeitsphase} betrachtet, die noch keinem Cluster zugeordnet sind
%		\item ausgehend von einem solchen Partikel werden seine Nachbarn auf ihren \wichtig{Tiefenwert} hin untersucht
%		\item der Nachbar mit dem höchsten Wert wird selektiert und es wird wiederrum der \wichtig{Distanzwert} von dessen Nachbarn betrachtet
%		\item diese Schleife läuft so lang, bis alle Nachbarn denselben oder einen geringeren \wichtig{Distanzwert} aufweisen (\liquidcenter{lokales Maximum} erreicht)
%		\item alle auf dem Weg selektierten Partikel werden anschließend einem (neuen oder vorhandenen) Cluster zugeordnet
%	\end{itemize}
%\end{frame}

%
% Zusammenhangskomponente vgl "Cluster innerhalb von Zusammenhangskomponenten"
% Gas wäre außen drumherum: Interessiert nicht.
% Große Zahl = Distanz, kleine = ID der Partikel
% 0 = Grenzfläche
% idealisiertes Beispiel (Distanz Gleitkommawert)
%
\begin{frame}{Zusammenhangskomponente}
	\includegraphics*[width=11cm]{media/cluster/cfd.png} \\
\end{frame}

\begin{frame}
	\includegraphics*[width=3.7cm]{media/cluster/cfd000.png} \\
	\includegraphics*[width=3.7cm]{media/cluster/cfd001.png} \\
	\includegraphics*[width=3.7cm]{media/cluster/cfd002.png} \\
	\includegraphics*[width=3.7cm]{media/cluster/cfd003.png}
\end{frame}

\begin{frame}
	\includegraphics*[width=3.7cm]{media/cluster/cfd000.png} \includegraphics*[width=3.7cm]{media/cluster/cfd004.png} \\
	\includegraphics*[width=3.7cm]{media/cluster/cfd001.png} \includegraphics*[width=3.7cm]{media/cluster/cfd005.png} \\
	\includegraphics*[width=3.7cm]{media/cluster/cfd002.png} \includegraphics*[width=3.7cm]{media/cluster/cfd006.png} \\
	\includegraphics*[width=3.7cm]{media/cluster/cfd003.png} \includegraphics*[width=3.7cm]{media/cluster/cfd007.png}
\end{frame}

\begin{frame}
	\includegraphics*[width=3.7cm]{media/cluster/cfd000.png} \includegraphics*[width=3.7cm]{media/cluster/cfd004.png} 	\includegraphics*[width=3.7cm]{media/cluster/cfd008.png} \\
	\includegraphics*[width=3.7cm]{media/cluster/cfd001.png} \includegraphics*[width=3.7cm]{media/cluster/cfd005.png} 	\includegraphics*[width=3.7cm]{media/cluster/cfd009.png} \\
	\includegraphics*[width=3.7cm]{media/cluster/cfd002.png} \includegraphics*[width=3.7cm]{media/cluster/cfd006.png} 	\includegraphics*[width=3.7cm]{media/cluster/cfd010.png} \\
	\includegraphics*[width=3.7cm]{media/cluster/cfd003.png} \includegraphics*[width=3.7cm]{media/cluster/cfd007.png}	\includegraphics*[width=3.7cm]{media/cluster/cfd011.png}
\end{frame}

%
% Schön zu sehen sind die gleiche Färbung nebeneinanderliegender Cluster
%
\begin{frame}{Überprüfung der erzeugten Cluster}
	Einfärbung der Partikel zur Clusterüberprüfung:
	\begin{itemize}
		\item zufällige Färbung in jedem Zeitschritt (links)
		\item Färbung anhand von Clusterparametern und
		\item Vererbung vom vorhergehenden Zeitschritt (rechts)
	\end{itemize}
	
	\includegraphics*[width=5.7cm]{media/eva/clusterfarbe-allRd-small.png}
	\includegraphics*[width=5.7cm]{media/eva/clusterfarbe-rdInteritance-small.png}
	
	Vererbung geschieht unter Nutzung des Clustervergleichs. % über den ich jetzt rede
\end{frame}

\section{Erkennung der Ereignisse}
%\frame{\tableofcontents[currentsection, hideothersubsections]}
\begin{frame}
	\includegraphics*[width=11.5cm]{media/VisMD-Ablaufschema}
	
	TODO: Attributen/UserInput nur bei Nachbar/CFD, SECC und Heuristik hervorheben
\end{frame}

\subsection{Clustervergleich (SECC)}

\begin{frame}
	Um Strukturereignisse zu erkennen, werden zwei aufeinanderfolgende Zeitschritte betrachtet und die Veränderung der Cluster zwischen diesen verfolgt. Anschließend werden daraus mit einer Heuristik Ereignisse abgeleitet.
\end{frame}

\begin{frame}{Clustervergleichsmatrix}

	TODO: Bild Clustervergleichsmatrix vorauss. 5x5 nötig:\\
	mit ClusterIDs und Kennzeichnung aktueller/vorhergehender Zeitschritt,\\
	mit Clustergröße,\\
	mit Partikelmengen

	%\item diese Zuordnung wird für zwei aufeinanderfolgende Zeitschritte betrachtet (aktueller und vorheriger Zeitschritt)
	%\item die Anzahl der Partikel pro Cluster wird in einer n-m-Matrix festgehalten, wobei n für die Anzahl an Clustern des aktuellen und m für die Clusteranzahl des vorhergehenden Zeitschritts steht
	%\item der Zeilenindex steht für die Cluster ID des aktuellen und der Spaltenindex für die Cluster ID des vorhergehenden Zeitschritts
	%\item die Zellen enthalten die zugehörige Partikelanzahl; semantisch ist dies die Anzahl gemeinsamer Partikel der Cluster zweier Zeitschritte
\end{frame}

\begin{frame}{Partnercluster}
	
	TODO: Bild Clustervergleichsmatrix:\\
	Spalten und Zeileneinfärbung (Rückwärts-/Vorwärtsgerichteter Vergleich)
	
	%\item die \wichtig{Clustervergleichsmatrix} wird sowohl Zeilen- als auch Spaltenweise betrachtet
	%\item dadurch kann das Verhältnis hinsichtlich der Partikelanzahl eines Clusters des aktuellen Zeitschrittes zu allen Clustern des vorhergehenden Zeitschrittes ermittelt werden (und umgekehrt)
	%\item diese zwei Richtungen werden vorwärtsgerichteter Vergleich (alter Zeitpunkt zu aktuellem) und rückwärtsgerichteter Vergleich (aktueller zu altem Zeitpunkt) genannt
	%\item Cluster mit gemeinsamen Partikeln werden als Partner bezeichnet
\end{frame}

\begin{frame}{Relevante Werte}
	TODO: Bild Clustervergleichsmatrix:\\
	Summen und Anteile in Extrazeile/Spalte außen dran\\
	Große Partner eventuell erst bei Split/Merge hervorheben (dort erst gebraucht!)
	
	Kann u.U. mit vorhergehender Folie verschmelzen (hängt von der Komplexität der Grafik ab).

	%\item Gesamtanzahl der Partner
	%\item Anzahl an großen Partnern. Große Partner sind solche Partnercluster mit vielen gemeinsamen Partikeln.
	%\item Grenzwert, ab wann ein Partner ein großer Partner ist. Der Grenzwert ist das Verhältnis der gemeinsamen Partikel des Partners im Vergleich zu der Gesamtzahl gemeinsamer Partikel des Clusters.
\end{frame}

\subsection{Ereignisheuristik}

\begin{frame}{Birth und Death}
	
	TODO: Bild Clustervergleichsmatrix:\\
	Spalte ohne gemeinsame Anteile\\
	Zeile ohne gemeinsame Anteile
	
	%\item Gesamtanzahl der Partner ist Null
	%\item Birth im rückwärtsgerichteten Vergleich: kein Partner des vorangegangenen Zeitschrittes enthält Partikel des derzeitigen Clusters
	%\item Death im vorwärtsgerichteten Vergleich: kein Partner des aktuellen Zeitschrittes enthält Partikel des vorangegangenen Clusters
\end{frame}

\begin{frame}{Birth und Death}
	EVTL GANZ RAUS!
	
	\begin{align*}
	p_{\text{gesamt}_r}, p_{\text{gesamt}_v} &\ldots \text{Gesamtzahl an Partnern (rückwärts- und vorwärtsgerichtet)}\\
	a_{\text{gesamt}_r}, a_{\text{gesamt}_v} &\ldots \text{Gesamtanteil gemeinsamer Partikel aller Partner}\\
	e_{bd} &\ldots \text{benutzerdefinierter Anteil an gemeinsamen Partikeln}
	\end{align*}
	\begin{equation}
	\begin{aligned}\label{eq:birth}
		p_{\text{gesamt}_r} &= 0 &(B)\\
		e_{bd} &\le a_{\text{gesamt}_r} &(C)
	\end{aligned}
	\end{equation}
	\begin{equation}
	\begin{aligned}\label{eq:death}
		p_{\text{gesamt}_v} &= 0 &(D)\\
		e_{bd} &\le a_{\text{gesamt}_v} &(E)
	\end{aligned}
	\end{equation}
	
	\wichtig{Birth} wird erkannt, wenn $(B \lor C) = \top$ und \wichtig{Death} wird erkannt, wenn $(D \lor E) = \top$.
\end{frame}

\begin{frame}{Merge und Split}
	
	TODO: Bild Clustervergleichsmatrix:\\
	Spalte mit großen Partnern und Grenzwert\\
	Spalte mit großen Partnern und Grenzwert
	
	%\item Mindestanzahl großer Partner
	%\item Grenzwert für diese Partner (ein Grenzwert für alle)
	%\item Merge im rückwärtsgerichteten Vergleich: mehrere große Partner des vorangegangenen Zeitschrittes sind zum derzeitigen Cluster verschmolzen
	%\item Split im vorwärtsgerichteten Vergleich: mehrere große Partner des aktuellen Zeitschrittes sind aus einem vorangegangenen Cluster hervorgegangen
\end{frame}

\begin{frame}{Merge und Split}
	EVTL GANZ RAUS!
	
	\begin{align*}
		p_{\text{groß}_r}, p_{\text{groß}_v} &\ldots \text{Anzahl großer Partner (rückwärts- und vorwärtsgerichtet)}\\
		a_{\text{partner}_r}, a_{\text{partner}_v} &\ldots \text{Anteil gemeinsamer Partikel eines Partners}\\
		e_{msA} &\ldots \text{benutzerdefinierter Anteil gemeinsamer Partikel}\\
		e_{msP} &\ldots \text{benutzerdefinierte Anzahl großer Partner}
	\end{align*}
	Inkrementierung von $p_{\text{groß}}$, wenn $a_{\text{partner}} \ge e_{msA}$.
	
	\begin{equation}
	\begin{aligned}\label{eq:merge}
	p_{\text{groß}_r} &\ge e_{msP} &(M)
	\end{aligned}
	\end{equation}
	\begin{equation}
	\begin{aligned}\label{eq:split}
	p_{\text{groß}_v} &\ge e_{msP} &(S)
	\end{aligned}
	\end{equation}
	
	\wichtig{Merge} wird erkannt, wenn $M = \top$ und \wichtig{Split} wird erkannt, wenn $S = \top$.
\end{frame}


\section{Visualisierung}
%\frame{\tableofcontents[currentsection, hideothersubsections]}
\begin{frame}
	\includegraphics*[width=11.5cm]{media/VisMD-Ablaufschema}
	
	TODO: Attributen/UserInput bei Nachbar/CFD, SECC/Heuristik, Visualisierung hervorheben
\end{frame}

\subsection{Strukturereignistaxonomie}

%
% numerische Werte und Kategorien mit vorstellen
%
\begin{frame}{Strukturereignistaxonomie}
	Ereignisparamenter sowie Einteilung möglicher visuellen Variablen für die Zuweisung der Parameter.

	\includegraphics*[width=11cm]{media/vis/strukturereignistaxonomie.png}
\end{frame}

\begin{frame}{Zuweisung visueller Attribute}
	Zuweisung von visuellen Attributen zu den Parametern
	\begin{itemize}
		\item Ort: Position im Raum (Aufgabenstellung)
		\item Zeitpunkt: Farbwert, Helligkeit, Füllstand (Textur)
		\item Art: abstrakt und narrativ symbolische Darstellung (Textur)
	\end{itemize}
\end{frame}

%
% Beispielhafte Zuweisung
% 3D-Darstellung schlecht:
% Die Erscheinung von Glyphen als Polygone ist blickwinkelabhängig (Form) und beleuchtungsabhängig (unterschiedliche Helligkeiten je nach Winkel der Flächen).
% Bewirkt Senkung der präattentiven Wirkung (kognitiver Aufwand). %präattentiv: Reiz erzeugen, der vom Nervensystem des Beobachters zwar wahrgenommen wird und einen Effekt erzeugt, jedoch nicht in das Bewusstsein dringt. Vorteil ist eine parallele, schnelle Verarbeitung (kürzer als 250	Millisekunden (ms)), die nicht gehemmt werden kann. Nachteilig ist, dass die Wahrnehmung	oberflächlich bleibt
% Erschwert Identifizierung bei Überlagerung mit der Partikeldarstellung (einzelne Polygone zwischen sehr vielen Kugeln) $\Rightarrow$ Wechsel zu 2D-Glyphen (Billboards).
%
\begin{frame}{Mockup mit beispielhafter Variablenzuweisung (WebGL)}
	\includegraphics*[height=6.9cm]{media/vis/MockupZufallsdatenOben45bunt.png}
\end{frame}

\subsection{Glyphdarstellung} %	Glyph ist die grafische Darstellung eines Schriftzeichens!

\begin{frame}{Glyphdarstellung Strukturereignis: Kodierung Ereignisart}
	Abstrakt symbolische Darstellung\\
					\includegraphics[width=.18\textwidth]{media/vis/GlyphEventTypeAbstractBirthWhite-bg}
					\hspace*{.06\textwidth}%
					\includegraphics[width=.18\textwidth]{media/vis/GlyphEventTypeAbstractDeathWhite-bg}
					\hspace*{.06\textwidth}%
					\includegraphics[width=.18\textwidth]{media/vis/GlyphEventTypeAbstractMergeWhite-bg}
					\hspace*{.06\textwidth}%
					\includegraphics[width=.18\textwidth]{media/vis/GlyphEventTypeAbstractSplitWhite-bg}
					
	Narrativ symbolische Darstellung\\
					\includegraphics[width=.18\textwidth]{media/vis/GlyphEventTypeBirthWhite-bg}
					\hspace*{.06\textwidth}%
					\includegraphics[width=.18\textwidth]{media/vis/GlyphEventTypeDeathWhite-bg}
					\hspace*{.06\textwidth}%
					\includegraphics[width=.18\textwidth]{media/vis/GlyphEventTypeMergeWhite-bg}
					\hspace*{.06\textwidth}%
					\includegraphics[width=.18\textwidth]{media/vis/GlyphEventTypeSplitWhite-bg}
					
	Der Rahmen erfüllt das Gesetz der Geschlossenheit. Seine Form hebt sich von der Kreisform ab (Kontrast zur Kugeldarstellung der Partikel) und gibt Hinweis auf exakte Ereignisposition.
\end{frame}

\begin{frame}{Glyphdarstellung Strukturereignis: Zeitdarstellung}
	Füllstand (Textur)\\
					\includegraphics[width=.14\textwidth]{media/vis/GlyphenEventTimeFill000}
					\hspace*{.1\textwidth}%
					\includegraphics[width=.14\textwidth]{media/vis/GlyphenEventTimeFill050}
					\hspace*{.1\textwidth}%
					\includegraphics[width=.14\textwidth]{media/vis/GlyphenEventTimeFill100}

	Helligkeit\\
					\includegraphics[width=.14\textwidth]{media/vis/GlyphenEventTimeBrightness000}
					\hspace*{.1\textwidth}%
					\includegraphics[width=.14\textwidth]{media/vis/GlyphenEventTimeBrightness050}
					\hspace*{.1\textwidth}%
					\includegraphics[width=.14\textwidth]{media/vis/GlyphenEventTimeBrightness100}

	Farbwert\\
					\includegraphics[width=.14\textwidth]{media/vis/GlyphenEventTimeHue000}
					\hspace*{.1\textwidth}%
					\includegraphics[width=.14\textwidth]{media/vis/GlyphenEventTimeHue050}
					\hspace*{.1\textwidth}%
					\includegraphics[width=.14\textwidth]{media/vis/GlyphenEventTimeHue100}
\end{frame}

\section{Ergebnisse}
\frame{\tableofcontents[currentsection, hideothersubsections]}

\subsection{Cluster- und Ereigniserkennung}
\begin{frame}
	Abbildung 7.3.
	Abbildung 7.4.
	Umfrage mit reinbringen, quantitatives nur kurz, Performance weglassen!
\end{frame}

\subsection{Visualisierung}
\begin{frame}
	Umfrageergebnisse, v.a. in situ
\end{frame}

\begin{frame}{Bewertung und Ausblick}
	Nötig?
	
	Erweiterung der Algos kann bei Ergebnisse kurz genannt werden
	Nutzung des Programms kann in Demo gezeigt werden
\end{frame}

\begin{frame}{Danksagung}
		Danksagung als Folie üblich?
\end{frame}

\end{document}





\begin{frame}{3D-Glyphen - Beleuchtung}
	\includegraphics*[height=6.9cm]{media/vis/Mockup-Zufallsdaten-persp-ArtForm-vornEntfernter.png}
\end{frame}

\begin{frame}{Zusammenfassung (wichtigste Schritte)}
	\begin{enumerate}
		\item Finden der Partikelnachbarn mittels \wichtig{kD-Baum}, Berücksichtigung periodischer Randbedingungen
		\item Cluster erstellen unter Nutzung der Nachbarn (\wichtig{CFD})
		\item Clustervergleich (\wichtig{SECC}) durch Nutzung der \wichtig{Clustervergleichsmatrix} und Erstellung von zwei Listen mit \wichtig{Partnerclustern} in Vorwärtsrichtung beziehungsweise rückwärtiger Richtung.
		\item Bestimmung der Ereignisse durch Anwendung von Verhältnisberechnungen auf diese Listen und Nutzung von benutzerdefinierten Grenzwerten für die \wichtig{Ereignisheuristik}.
	\end{enumerate}
\end{frame}

\begin{frame}{Alle Cluster- und Ereigniserkennungsschritte}
	\begin{enumerate}
		\item
		\begin{enumerate}
			\item Aufbau der Partikelliste mit den Daten des MultiParticleDataCalls.
			\item Erstellung von \wichtig{kD-Bäumen} für die Nachbarschaftserkennung.
			\item Nutzung des \wichtig{kD-Baum} Suchalgorithmus der \wichtig{ANN-Bibliothek}, um Nachbarn jedes Partikels, unter Berücksichtigung des \wichtig{Partikelradiusmultiplikator}, zu diesem hinzuzufügen.
		\end{enumerate}
		\item
		\begin{enumerate}
			\item Cluster erstellen unter Nutzung der Nachbarn.
			\item Reduktion der \wichtig{Mindestgrößencluster} innerhalb von Zusammenhangskomponenten, die weniger Partikel besitzen als für die benutzerdefinierte minimale \wichtig{Clustergröße} erlaubt.
		\end{enumerate}
		\item Clustervergleich (\wichtig{SECC}) durch Nutzung der \wichtig{Clustervergleichsmatrix} und Erstellung von zwei Listen mit \wichtig{Partnerclustern} in Vorwärtsrichtung beziehungsweise rückwärtiger Richtung.
		\item Bestimmung der Ereignisse durch Anwendung von Verhältnisberechnungen auf diese Listen und Nutzung von benutzerdefinierten Grenzwerten für die \wichtig{Ereignisheuristik}.
	\end{enumerate}
\end{frame}

%%%
%%% Strukturereignisse Zwischenverteidigung
%%%

\section{Erkennung der Strukturereignisse}
\frame{\tableofcontents[currentsection, hideothersubsections]}

\begin{frame}
	Durch Vergleich der Partikelzugehörigkeit zu Clustern über zwei Frames. Eventuell ein Vergleich von Cluster IDs bei Vorhandensein von Clusterlisten im vorhergehenden Schritt.
	\begin{itemize}
		\item Birth: Partikel hat neue Cluster ID und Cluster existierte vorher nicht
		\item Death: Partikel hat keine Cluster ID, hatte vorher eine und dieser Cluster existiert nicht mehr
		\item Merge: Partikel hat neue Cluster ID und alter Cluster existiert noch
		\item Split: Partikel hat neue Cluster ID und alter Cluster existiert noch
	\end{itemize}
	Falls dem Cluster bereits ein Ereignis zugewiesen wurde, kein neues Ereignis erstellen.
\end{frame}
