\documentclass[10pt]{beamer}
\usepackage[utf8x]{inputenc}
\usepackage[ngerman]{babel}
\usepackage{amsmath}

\usepackage{verbatim}

\usepackage{multicol} %Spalten

\setlength{\parskip}{0.5em} % Abstand nach Absatz

\usetheme{Dresden} %http://www.hartwork.org/beamer-theme-matrix/
%\usetheme{Goettingen} 
\usecolortheme{seahorse} %http://www.hartwork.org/beamer-theme-matrix/
\setbeamercovered{transparent}

%%%
% Colors
%
\definecolor{orange}{rgb}{.99,.78,0}
\definecolor{blau}{rgb}{0,.52,.73}
\definecolor{blaurot}{rgb}{.8,.52,.73}
\newcommand{\liquid}[1]{\textcolor{blau}{#1}}
\newcommand{\liquidcenter}[1]{\textcolor{blaurot}{#1}}
\newcommand{\wichtig}[1]{\textit{#1}}
\newcommand{\gas}[1]{\textcolor{orange}{#1}}

%%%
% Quellcode
%
\usepackage{xcolor}
\usepackage{listings}

\definecolor{dkgreen}{rgb}{0,0.6,0}
\definecolor{mygray}{rgb}{0.5,0.5,0.5}
\definecolor{mymauve}{rgb}{0.58,0,0.82}
\lstset{ %
	backgroundcolor=\color{white},   % choose the background color; you must add \usepackage{color} or \usepackage{xcolor}
	basicstyle=\scriptsize,        % the size of the fonts that are used for the code
	breakatwhitespace=false,         % sets if automatic breaks should only happen at whitespace
	breaklines=false,                 % sets automatic line breaking
	captionpos=b,                    % sets the caption-position to bottom
	commentstyle=\color{dkgreen},    % comment style
	escapeinside={\%*}{*)},          % if you want to add LaTeX within your code
%	frame=single,                    % adds a frame around the code
	keepspaces=true,                 % keeps spaces in text, useful for keeping indentation of code (possibly needs columns=flexible)
	keywordstyle=\color{blue},       % keyword style
	numbers=left,                    % where to put the line-numbers; possible values are (none, left, right)
	numbersep=5pt,                   % how far the line-numbers are from the code
	numberstyle=\tiny\color{mygray}, % the style that is used for the line-numbers
	rulecolor=\color{black},         % if not set, the frame-color may be changed on line-breaks within not-black text (e.g. comments (green here))
	showspaces=false,                % show spaces everywhere adding particular underscores; it overrides 'showstringspaces'
	showstringspaces=false,          % underline spaces within strings only
	showtabs=false,                  % show tabs within strings adding particular underscores
	stepnumber=2,                    % the step between two line-numbers. If it's 1, each line will be numbered
	stringstyle=\color{mymauve},     % string literal style
	tabsize=2,                       % sets default tabsize to 2 spaces
	title=\lstname                   % show the filename of files included with \lstinputlisting; also try caption instead of title
}

%%%
% Wide frames (low margin)
%
\usepackage{calc}
\usepackage{environ}

\newcommand{\halfmargin}{0.05\paperwidth}
\newcommand{\margin}{0.10\paperwidth}

%\beamersetrightmargin{\margin}
%\beamersetleftmargin{\margin}

\NewEnviron{wideframe}[1][]{%
	\begin{frame}{#1}
		\makebox[\textwidth][c]{
			\begin{minipage}{\dimexpr\paperwidth-\halfmargin}
				\BODY
			\end{minipage}}
		\end{frame}
	}


%%%
% Page number
%
\addtobeamertemplate{navigation symbols}{}{%
	\usebeamerfont{footline}%
	\usebeamercolor[fg]{footline}%
	\hspace{1em}%
	\insertframenumber/\inserttotalframenumber
}

%%%
% Title
%
\title[Visualisierung von Strukturveränderungen in Molekulardynamikdaten]{Visualisierung von Strukturveränderungen in Molekulardynamikdaten}
\subtitle[Bachelorarbeit Verteidigung]{Bachelorarbeit Verteidigung}
\author[Richard Hähne]{Richard Hähne}
\institute[Professur für Computergraphik und Visualisierung]{Institut für Software- und Multimediatechnik\\Professur für Computergraphik und Visualisierung}
%\institute[Fakultät Informatik – Institut SMT - Professur für Computergraphik und Visualisierung]{Institut für Software- und Multimediatechnik\\Professur für Computergraphik und Visualisierung}
% logo of my university
\titlegraphic{\includegraphics[height=.65cm]{mediaDefense/3rdparty/tulogosw}%
	\hspace{1.55cm}%
	\includegraphics[height=.65cm]{mediaDefense/3rdparty/computergrafikVisualisierungLogo}%
	\hspace{1.55cm}%
	\includegraphics[height=.6cm]{mediaDefense/3rdparty/MegaMol_Logo}
	}
\date{\today}
 
\begin{document}
\maketitle
%\frame{\tableofcontents[hideallsubsections]}

\section{Einleitung}
%\frame{\tableofcontents[currentsection, hideothersubsections]}

\subsection{Strukturereignisse während einer Molekulardynamiksimulation}

%
% Diffusion in dünnen Schichten bei Temperatureinfluss (Eigenschaftsveränderung der Oberfläche (Farbe, Härte etc), Haftung)
% mikroskopische Abtragung der Beschichtung von Turbinenblättern durch lokale Druckveränderungen
%
\begin{frame}[<+->]{Motivation}
	\begin{itemize}
		\item durch Molekulardynamik können Vorhersagen über Materialverhalten getroffen werden
		\item dazu werden meist Atome und Moleküle (\wichtig{Partikel}) begrenzter Zahl über kurze Zeiträume betrachtet
		\item durch räumliche Nähe der \wichtig{Partikel} bilden sich Strukturen
		\item hier: Phasenübergang zwischen \gas{Gas} und \liquid{Flüssigkeit} begrenzt Strukturen
		\item Partikel bewegen sich, Strukturen werden verändert
		\item daher ist es sinnvoll, Strukturveränderungen zu erkennen
	\end{itemize}
\end{frame}

%\includegraphics*<1->[width=\textwidth]{mediaDefense/SignedDistanceColor-Show-Frame-60}

%
%	Datensatz:
%	- Flüssigkeitsfilm umgeben von Vakuum
%	- expandiert rapide (zerreißt)
%	- es entsteht eine Gasphase (Partikel lösen sich aus der Flüssigkeit)
%	- zwei Fronten: dazwischen Filamente und Tröpfchen
%	- dargestellt durch eine Animation mit 150 Zeitschritten!
%	- Strukturbestimmend sind die Position und der Abstand zur Phasengrenze!
%	Ziel:
%	- interessant zu wissen, was währenddessen mit der Struktur passiert
%	-> Strukturereignisse
%
\begin{frame}
	\begin{columns}[T]
		\begin{column}{5.5cm}
			Zerreißender Flüssigkeitsfilm:\\
			\visible<1->{
				{\scriptsize $t_S=0$}\\
				\includegraphics*[width=5.5cm]{mediaDefense/SignedDistanceColor-Frame-000-letterbox.png}
			}\\

			\visible<2->{
				{\scriptsize $t_S=20$}\\		
				\includegraphics*[width=5.5cm]{mediaDefense/SignedDistanceColor-Frame-020-letterbox.png}
			}\\

			\visible<3->{
				{\scriptsize $t_S=30$}\\		
				\includegraphics*[width=5.5cm]{mediaDefense/SignedDistanceColor-Frame-030-letterbox.png}
			}\\
			
			\visible<4->{
				{\scriptsize $t_S=100$}\\		
				\includegraphics*[width=5.5cm]{mediaDefense/SignedDistanceColor-Frame-100-letterbox.png}
			}\\
		\end{column}
		\begin{column}{6.5cm}
			\only<1-5>{
				\begin{itemize}
					\item Flüssigkeitsfilm umgeben von Vakuum
					\item<2-> zerreißt
					\item<2-> zwei Phasen
					\begin{itemize}
						\item \liquid{Flüssigphase} (blau - lila)
						\item \gas{Gasphase}
					\end{itemize}
					\item<3-> Entstehung von Filamenten
					\item<4-> Entstehung von Tröpfchen
				\end{itemize}
				\onslide<5->{Datensatz}
				\begin{itemize}
					\item<5-> zwei Millionen Partikel
					\item<5-> 150 Zeitschritte
					\item<5-> Partikelposition je Zeitschritt
					\item<5-> Partikelabstand zur am nächsten gelegenen Phasengrenze % (lila Partikel besitzen größten Abstand)
				\end{itemize}
			}
			\only<6->{
				Auftretende \wichtig{Strukturereignisse}:\\
				\includegraphics*[width=6.5cm]{mediaDefense/Strukturereignisse.png}
			}
		\end{column}
	\end{columns}
\end{frame}

%TODO evtl Folie mit Video

%
% Lösung = Aufgabenstellung
%
\begin{frame}{Ziel}
	Wie verhält sich die Flüssigkeit, welche Strukturveränderungen treten auf?
	
	Die vorliegende Animation ist zur Analyse nur bedingt geeignet:
	\begin{itemize}
		\item vieles geschieht gleichzeitig: Erfassung schwierig
		\item sequentieller Ablauf: quantitativer Vergleich schwierig
		%\item Diagramm: unübersichtlich bei vielen und übereinanderlagernden Daten
	\end{itemize}
	Lösung: Darstellung der zeitlichen Entwicklung der Strukturveränderungen im geometrischen Kontext der Originaldaten.
\end{frame}

%
% Abbildung eines Ergebnisses der Arbeit
%
\begin{frame}{Ziel}
	\includegraphics*[width=\textwidth]{mediaDefense/ereigniserkennung/5r-10-ms2-40-bd2-allRd02-shot-f80-2-brightness-small}
\end{frame}

\subsection{Erkennung und Visualisierung der Strukturereignisse}

%
% Wie lassen sich diese Strukturereignisse erkennen und darstellen?
% Partikeldaten: nur die hier relevanten
% Vorzeichenbehaftete Distanz = Abstand zum Phasenübergang
% Gliederung des Vortrags entspricht der Sequenz der Berechnungsschritte
%
% 1) Analyse der Struktur: Analyse der Partikelstruktur für jeden Zeitschritt
% 2) Erkennung von Veränderungen: Erkennung von Änderungen an der Struktur von einem zum nächsten Zeitschritt (Strukturereignisse)
% 3) Darstellung: Anzeige dieser Ereignisse im Ortsraum der Partikel
\begin{wideframe}
	\includegraphics*[width=\textwidth]{mediaDefense/VisMD-Ablaufschema-EinAusgang}
\end{wideframe}

%
% Erinnerung: Strukturen bestimmt durch die Position und den Abstand zur Phasengrenze
% Es verlangsamt die Entwicklung, wenn ich sehr lang auf die Ergebnisse warten muss.
%
\begin{frame}{Herausforderungen}
	\begin{itemize}
		\item Sinnvolle Einteilung der Partikel in Strukturen, um Ereignisse erkennen zu können.
		\item Vorgehen, um Veränderungen der Struktur über die Zeit verfolgen zu können.
		\item Berechnungsdauer überschaubar halten (Überprüfung der Algorithmen).
		\item Darstellung der Ereignisse.
	\end{itemize}
\end{frame}

%
% An den Berechnungsschritten und Herausforderungen ist ersichtlich, dass sich die Arbeit mehreren Themenbereichen zuordnen lässt:
% •	Strukturanalyse VOR ALLEM
% •	schauen, was in der Molekulardynamik üblich ist
% •	Visualisierungsmethoden im Allgemeinen
%
% hier im Vortrag Konzentration auf Strukturanalyse
%
% Skelettext: Oberflächen, Volumen, Punktdaten: umfangreiche Pre- oder Postprocessingverfahren zum Entfernen von Artefakten
%
% Konturbäume: Datenelemente anhand ihres Funktionswertes gruppieren und für verschiedene Funktionswerte Zusammenhangskomponenten erkennen
% basierend auf Skelettextraktion (Nachbarschaftsbildung über Reichweite: Klasing et al.)
% basierend auf Konturbaum: Nutzung kritischer Funktionswert im CFD: Carr et al.
%
% Einordnung dieser Arbeit:
% - Strukturuntersuchung von Punktdaten
% - Element- und Zeitdarstellung in Molekulardynamiksimulationen
% ALT:
% - Visualisierung (zeitabhängiger) Elemente im Ortsraum visualisierter Punktdaten
% - Verfolgung zeitveränderlicher Daten ohne Baumstrukturen
%
\begin{frame}{Verwandte Arbeiten}
	\begin{itemize}
		\item<1-> Strukturanalyse räumlicher Datensätze
		\begin{itemize}
			\item Skelettextraktion
			\item Konturbäume (Funktionswert, Zusammenhangskomponenten)
		\end{itemize}
		\item<2-> Element- und Zeitdarstellung in Molekulardynamiksimulationen
		\begin{itemize}
			\item Elemente sind meist dreidimensional, farbkodiert
			\item Zeitdarstellung durch Animation oder Diagramme
		\end{itemize}
		\item<3-> Visualisierung
		\begin{itemize}
			\item Anordnung und Filterung von Datenobjekten
			%\item Anordnung (Gleicher et al.) und Filterung von Datenobjekten (Lima)
			\item Glyphdesign
		\end{itemize}
	\end{itemize}
\end{frame}

%
% Arbeiten aus der Strukturanalyse
%
% 1:
% Vulkansee mit Insel in der Mitte
% Kraterrand, Berg, Tal haben eine Höhe: Höhendaten = Skalarfeld über dem Gebiet, den Orten auf der Karte
% anhand der Höhe werden die Orte auf der Karte in einen Baum überführt:
% kritische Punkte wie Maxima, Minima, Sattelpunkte repräsentieren Knoten
% Höhe im Baum gestrichelte Linien
% Besonderheit: liegt ein Tal zwischen zwei Bergkuppen, dann befinden sich diese Bergkuppen in separaten Ästen
% Kraterrand A und Berg B sind getrennt durch Tal D
% A und B sind also für die Höhe >12 und auch noch Höhe >8 separate Zusammenhangskomponenten
% Übertragung auf die Partikeldaten heißt: Zusammenhangskomponenten durch Phasenübergang Gasphase <-> Flüssig
% Strukturanalyse daraus vereinfacht abgeleitet: Nutzung Distanzwert zum Phasenübergang: Beschränkung auf Maxima
%
% Oberflächenrelief eines Vulkankratersees mit einer Insel in der Mitte (links) sowie der zugehörige Konturbaum (rechts). A: Maximum der Kraterkante; B: Maximum des inneren Berges; C und D: Sattelpunkte; E,F: Minima
% Bild Höhendaten: Ablaufen der Daten nach der Höhe => komme ich an ein Maximum oder Minimum, bin ich fertig (A,B,E,F). Sattelpunkte teilen den Baum (C,D)
%
%
% 2: Temporal
% Konturbäume auch zur temporalen Verfolgung
% Weiterer Graph dazu (Änderungsgraph)
% farbie Linien zeigen Datenpunkte mit - Übertragung auf Partikeldatensatz - bestimmter Distanzwerte: sagen wir den Wert 0 für die Phasengrenze
% jede Farbe schließt eine Zusammenhangskomponente ein
% Zeitverfolgung über Volumen => hier Partikelmenge statt Volumen
%
% Evolution of three isosurfaces. An intersection point on an edge of a contour tree represents a contour component. (a) By using precomputed correspondence information in time-dependent contour trees and (b) a graph representing the topology changes of time-varying isosurfaces is immediately constructed for any selected isovalue. (c) Seed sets generated from the contour trees are used for rapid extraction of the time-dependent surfaces of segmented contours. Note that colors represent correspondence among intersection points in contour trees, nodes of the graph, and contour components.
% Achtung (b) hat nichts mit Konturbäumen zu tun (Verfolgung über die Zeit anstatt über Skalarfeldwerten)
%
\begin{frame}{Verwandte Arbeiten: Strukturanalyse: Konturbäume}
	\begin{columns}[T]
		\begin{column}{.5\textwidth}
			\scriptsize {
				H. Carr et al.: \textit{Flexible isosurfaces: Simplifying and displaying scalar topology using the contour tree.} (2010)
			
				\begin{itemize}
					\item ein Skalarfeld (links) wird in einen Baum überführt durch Beachtung kritischer Punkte (Konturbaum, rechts)
					\item Selektion von Zusammenhangskomponenten %aus Niveaumengen (Datenpunkte mit festgelegter Höhe)
					%\item Selektion bestimmter Niveaumengen und Zusammenhangskomponenten
				\end{itemize}
			}
		\end{column}
		\begin{column}{.5\textwidth}
			\includegraphics*[width=\textwidth]{mediaDefense/3rdparty/Carr2010FlexibleIsosurfacesScalarContourTree}
		\end{column}
	\end{columns}
	\vspace{.08\textheight}
	\visible<2>{
		\begin{columns}[T]
			\begin{column}{.5\textwidth}
				\scriptsize {
					B.-S. Sohn und C. Bajaj: \textit{Time-Varying Contour Topology.} (2006)

					\begin{itemize}
						\item zeitliche Änderung wird in topologischem Änderungsgraph gespeichert (b)
						\item Nutzung von (b) in Verbindung mit Konturbaum - Veränderung von Zusammenhangskomponenten verfolgbar (c)
						\item zeitlicher Vergleich durch Prüfung der sich überschneidenden Volumen
					\end{itemize}
				}
			\end{column}
			\begin{column}{.5\textwidth}
				\includegraphics*[width=\textwidth]{mediaDefense/3rdparty/Sohn2006Time-VaryingContourTopology}
			\end{column}
		\end{columns}
	}
\end{frame}


\section{Analyse der Partikelstruktur}
%\frame{\tableofcontents[currentsection, hideothersubsections]}

\subsection{Anforderungen}

%	- Begriff CLUSTER muss hier fallen! -> Einteilung der Partikel in Cluster
\begin{wideframe}
	\includegraphics*[width=\textwidth]{mediaDefense/VisMD-Ablaufschema-Strukturanalyse}
\end{wideframe}

\begin{frame}Cluster als Zusammenhangskomponenten
	\includegraphics*[width=11cm]{mediaDefense/cluster/SignedDistanceColor-Frame-090-clipplane20-letterbox.png}\\
	\visible<2>{\includegraphics*[width=11cm]{mediaDefense/cluster/SignedDistanceColor-Frame-090-clipplane20-letterbox-Cluster.png}}
%	Einteilung der Partikelmenge in Cluster!
\end{frame}

\begin{frame}Cluster innerhalb von Zusammenhangskomponenten
	\includegraphics*[width=11cm]{mediaDefense/cluster/SignedDistanceColor-Frame-016-clipplane20-letterbox.png}\\
	\visible<2>{\includegraphics*[width=11cm]{mediaDefense/cluster/SignedDistanceColor-Frame-016-clipplane20-letterbox-Cluster.png}}
\end{frame}

\subsection{Cluster Fast Depth (CFD) Algorithmus}
%
% Zusammenhangskomponente vgl "Cluster innerhalb von Zusammenhangskomponenten"
% Gas wäre außen drumherum: Interessiert nicht.
% Große Zahl = Distanz, kleine = ID der Partikel
% 0 = Grenze des Phasenübergangs: geschlossen, über nutzerdefinierten Wertebereich für die lokale Dichte bestimmt
% idealisiertes Beispiel (Distanz Gleitkommawert)
%
\begin{frame}{Zusammenhangskomponente}
	\includegraphics*[width=11cm]{mediaDefense/cluster/cfd.png} \\
\end{frame}

%
% alle Partikel werden sequentiell durchlaufen
% dabei werden \gas{Gaspartikel} ignoriert und nur Partikel der {Flüssigkeitsphase} betrachtet, die noch keinem Cluster zugeordnet sind
% ausgehend von einem solchen Partikel werden seine Nachbarn auf ihren {Distanzwert} hin untersucht
% der Nachbar mit dem höchsten Wert wird selektiert und es wird wiederrum der {Distanzwert} von dessen Nachbarn betrachtet
% diese Schleife läuft so lang, bis alle Nachbarn denselben oder einen geringeren {Distanzwert} aufweisen ({lokales Maximum} erreicht)
% alle auf dem Weg selektierten Partikel werden anschließend einem (neuen oder vorhandenen) Cluster zugeordnet
%
% Bei 1-3: Partikel 6 wird gewählt, überspringen einiger Schritte (analog): 2-3
% 2-3 Pfad ist rotmarkiert und da 77 schon Cluster hat, werden die markierten Partikel hinzugefügt: 2-4
% 2-4: Partikel 9 wird gewählt, analog
% 3-3: Aufgrund der Abbruchbedingung schon bei 42 schluss: zwei Cluster
%
% < statt <= als Abbruchbedingung, da Rechenaufwand u.U. lang und Vorteil Zwischencluster, der aber eher unerwünschte Ergebnisse liefern könnte -> aus Designsicht doofe Entscheidung
%
\begin{frame}
%	\visible<1->{\includegraphics*[width=.08\textwidth]{mediaDefense/cluster/cfd.png}}
	\visible<2->{\includegraphics*[width=.08\textwidth]{mediaDefense/cluster/cfd000.png}}
	\visible<3->{\includegraphics*[width=.08\textwidth]{mediaDefense/cluster/cfd001.png}}
	\visible<4->{\includegraphics*[width=.08\textwidth]{mediaDefense/cluster/cfd002.png}}
	\visible<5->{\includegraphics*[width=.08\textwidth]{mediaDefense/cluster/cfd003.png}
		\includegraphics*[width=.08\textwidth]{mediaDefense/cluster/cfd004.png}
		\includegraphics*[width=.08\textwidth]{mediaDefense/cluster/cfd005.png}
	}
	\visible<6->{\includegraphics*[width=.08\textwidth]{mediaDefense/cluster/cfd006.png}}
	\visible<7->{\includegraphics*[width=.08\textwidth]{mediaDefense/cluster/cfd007.png}
		\includegraphics*[width=.08\textwidth]{mediaDefense/cluster/cfd008.png}
		\includegraphics*[width=.08\textwidth]{mediaDefense/cluster/cfd009.png}
	}
	\visible<8->{\includegraphics*[width=.08\textwidth]{mediaDefense/cluster/cfd010.png}}
	
	\only<1>{\includegraphics*[width=\textwidth]{mediaDefense/cluster/cfd000.png}}
	\only<2>{\includegraphics*[width=\textwidth]{mediaDefense/cluster/cfd001.png}}
	\only<3>{\includegraphics*[width=\textwidth]{mediaDefense/cluster/cfd002.png}}
	\only<4>{\includegraphics*[width=\textwidth]{mediaDefense/cluster/cfd003.png}}
	\only<5>{\includegraphics*[width=\textwidth]{mediaDefense/cluster/cfd006.png}}
	\only<6>{\includegraphics*[width=\textwidth]{mediaDefense/cluster/cfd007.png}}
	\only<7>{\includegraphics*[width=\textwidth]{mediaDefense/cluster/cfd010.png}}
	\only<8>{\includegraphics*[width=\textwidth]{mediaDefense/cluster/cfd011.png}}
\end{frame}

%
% Cluster in Filamenten
% Farbe ändert sich => noch keine Zusammenhänge zwischen den Zeitschritten
%
\begin{frame}{Überprüfung der erzeugten Cluster}
	Einfärbung der Partikel zur Clusterüberprüfung:
	\begin{itemize}
		\item zufällige Färbung in jedem Zeitschritt
		\item es existieren keine Zusammenhänge zwischen den Zeitschritten
	\end{itemize}
	%\includegraphics*[width=5.7cm]{mediaDefense/eva/clusterfarbe-allRd-small.png}
	%\includegraphics*[width=5.7cm]{mediaDefense/eva/clusterfarbe-rdInteritance-small.png}
	
	{\scriptsize $t_S=63$}
	\hspace{.41\textwidth}
	{\scriptsize $t_S=64$}
	
	\includegraphics*[width=.47\textwidth]{mediaDefense/cluster/5r-10ms2-40bd2allRd02_clusterfarben_f63_crop}%
	\hspace{.04\textwidth}%
	\includegraphics*[width=.47\textwidth]{mediaDefense/cluster/5r-10ms2-40bd2allRd02_clusterfarben_f64_crop}
\end{frame}

\section{Erkennung der Ereignisse}
%\frame{\tableofcontents[currentsection, hideothersubsections]}

\subsection{Clustervergleich (SECC)}

%
% SECC: Vergleich von Clustern zweier aufeinanderfolgender Zeitschritte
% Heuristik zur Ableitung von Ereignissen
%
\begin{wideframe}
	\includegraphics*[width=\textwidth]{mediaDefense/VisMD-Ablaufschema-Ereigniserkennung}
\end{wideframe}

%
% 1: vor 
%
% 2:
% zwei aufeinanderfolgende Zeitschritte
% die Farben entsprechen der zufälligen Clustereinfärbung
% 
%
\begin{frame}
	\visible<2>{\includegraphics*[height=.48\textheight]{mediaDefense/ereigniserkennung/cluster-ti-1}}\\
	\includegraphics*[height=.48\textheight]{mediaDefense/ereigniserkennung/cluster-ti}
\end{frame}

%
% Alter Zeitschritt
% CLUSTERGRÖßE: Anzahl enthaltener Partikel
% aktueller analog (drei Cluster): Zusammenhang durch GEMEINSAME PARTIKEL
%
\begin{wideframe}Clustervergleichsmatrix
	
	\only<1>{\includegraphics*[width=\textwidth]{mediaDefense/ereigniserkennung/ereigniserkennung-clustervergleichsmatrix-ti-1}}
	\only<2>{\includegraphics*[width=\textwidth]{mediaDefense/ereigniserkennung/ereigniserkennung-clustervergleichsmatrix-ti-1-clustergroesze}}
	\only<3>{\includegraphics*[width=\textwidth]{mediaDefense/ereigniserkennung/ereigniserkennung-clustervergleichsmatrix-ti}}
	\only<4>{\includegraphics*[width=\textwidth]{mediaDefense/ereigniserkennung/ereigniserkennung-clustervergleichsmatrix-ti-clustergroesze}}
\end{wideframe}

%
% Gemeinsame Partikel: PARTNERCLUSTER, kurz PARTNER
% Verhältnisse: gemeinsame Partikel zu Clustergröße, separat für beide Zeitschritte
% Prozentangabe: übersichtlicher, nun Werte anschauen
%
\begin{wideframe}Partnercluster
	
	\only<1>{\includegraphics*[width=\textwidth]{mediaDefense/ereigniserkennung/ereigniserkennung-partnercluster-gemPartikel}}
	\only<2>{\includegraphics*[width=\textwidth]{mediaDefense/ereigniserkennung/ereigniserkennung-partnercluster-verhaeltnis}}
	\only<3>{\includegraphics*[width=\textwidth]{mediaDefense/ereigniserkennung/ereigniserkennung-partnercluster-prozent}}
\end{wideframe}

\subsection{Ereignisheuristik}

\begin{wideframe}Ereigniserkennung: Death
	
	\only<1>{\includegraphics*[width=\textwidth]{mediaDefense/ereigniserkennung/ereigniserkennung-heuristik-death1}}
	\only<2>{\includegraphics*[width=\textwidth]{mediaDefense/ereigniserkennung/ereigniserkennung-heuristik-death2}}
\end{wideframe}

\begin{wideframe}Ereigniserkennung: Merge
	
	\only<1>{\includegraphics*[width=\textwidth]{mediaDefense/ereigniserkennung/ereigniserkennung-heuristik-merge1}}
	\only<2>{\includegraphics*[width=\textwidth]{mediaDefense/ereigniserkennung/ereigniserkennung-heuristik-merge2}}
\end{wideframe}

\begin{wideframe}Ereigniserkennung: Fehldetektion (Split)
	
	\only<1>{\includegraphics*[width=\textwidth]{mediaDefense/ereigniserkennung/ereigniserkennung-heuristik-split1}}
	\only<2>{\includegraphics*[width=\textwidth]{mediaDefense/ereigniserkennung/ereigniserkennung-heuristik-split2}}
\end{wideframe}

\begin{wideframe}Ereigniserkennung: Zusammenfassung
	
	\only<1>{\includegraphics*[width=\textwidth]{mediaDefense/ereigniserkennung/ereigniserkennung-richtung-vorwaerts}}
	\only<2>{\includegraphics*[width=\textwidth]{mediaDefense/ereigniserkennung/ereigniserkennung-richtung-rueckwaerts}}
\end{wideframe}

\section{Visualisierung}
%\frame{\tableofcontents[currentsection, hideothersubsections]}

\subsection{Grundlagen}
\begin{wideframe}
	\includegraphics*[width=\textwidth]{mediaDefense/VisMD-Ablaufschema-Visualisierung}
\end{wideframe}

\begin{frame}{Glyphen in situ}
	\includegraphics*[width=\textwidth]{mediaDefense/ereigniserkennung/5r-10-ms2-40-bd2-allRd02-shot-f80-hue-small}
\end{frame}

%
% Ereignisparameter
% Einteilung visueller Variablen
% numerische Werte (metrische Skala) und Kategorien (Nominalskala)
%
\begin{frame}{Ereignisparameter und visuelle Variablen}
	\includegraphics*[width=\textwidth]{mediaDefense/vis/strukturereignistaxonomie}
\end{frame}

%
% Position im Raum (Aufgabenstellung)!
% Zeitpunkt: Farbwert, Helligkeit, Füllstand (Textur)
% Art: abstrakt und narrativ symbolische Darstellung (Textur)
%
\begin{frame}{Zuweisung der visuellen Variablen}
	\includegraphics*[width=\textwidth]{mediaDefense/vis/strukturereignistaxonomie-zuweisung}
\end{frame}

%
% Beispielhafte Zuweisung
% 3D-Darstellung schlecht:
% Die Erscheinung von Glyphen als Polygone ist blickwinkelabhängig (Form) und beleuchtungsabhängig (unterschiedliche Helligkeiten je nach Winkel der Flächen).
% Bewirkt Senkung der präattentiven Wirkung (kognitiver Aufwand). %präattentiv: Reiz erzeugen, der vom Nervensystem des Beobachters zwar wahrgenommen wird und einen Effekt erzeugt, jedoch nicht in das Bewusstsein dringt. Vorteil ist eine parallele, schnelle Verarbeitung (kürzer als 250 Millisekunden (ms)), die nicht gehemmt werden kann. Nachteilig ist, dass die Wahrnehmung oberflächlich bleibt
% Erschwert Identifizierung bei Überlagerung mit der Partikeldarstellung (einzelne Polygone zwischen sehr vielen Kugeln) $\Rightarrow$ Wechsel zu 2D-Glyphen (Billboards).
%
\begin{frame}{Mockup mit beispielhafter Variablenzuweisung (WebGL)}
	\includegraphics*[height=6.9cm]{mediaDefense/vis/MockupZufallsdatenOben45bunt.png}
\end{frame}

\subsection{Glyphdarstellung} %	Glyph ist die grafische Darstellung eines Schriftzeichens!

%
% Förderung präattentive Wahrnehmung: möglichst einfach gestaltet, wenig Variablen
% -> unterstützt durch Gesetz der Geschlossenheit
% gutes Rand-Flächen-Verhältnis, um Platz für Symbolik innerhalb zu haben => Form sollte wenig spitze Winkel besitzen => Kreis bietet optimales Verhältnis, hebt sich aber zu wenig ab
%
\begin{frame}{Glyphdarstellung Strukturereignis: Kodierung Ereignisart}
	\begin{columns}[T]
		\begin{column}{.5\textwidth}
			\only<1>{\includegraphics[width=.5\textwidth]{mediaDefense/vis/GlyphRahmenKreis}}
			\only<2>{\includegraphics[width=.5\textwidth]{mediaDefense/vis/GlyphRahmenOhnePos}}
			\only<3-4>{\includegraphics[width=.5\textwidth]{mediaDefense/vis/GlyphRahmen}}
			\only<4>{
				
				\vspace{1mm}%
				\hspace{.2\textwidth}%
				\includegraphics[width=.1\textwidth]{mediaDefense/vis/Partikel-blau}}
			\only<5->{
				Narrativ symbolische Darstellung\\
				\includegraphics[width=.22\textwidth]{mediaDefense/vis/GlyphEventTypeBirthWhite-bg}
				\hspace*{.04\textwidth}%
				\includegraphics[width=.22\textwidth]{mediaDefense/vis/GlyphEventTypeDeathWhite-bg}
				\hspace*{.04\textwidth}%
				\includegraphics[width=.22\textwidth]{mediaDefense/vis/GlyphEventTypeMergeWhite-bg}
				\hspace*{.04\textwidth}%
				\includegraphics[width=.22\textwidth]{mediaDefense/vis/GlyphEventTypeSplitWhite-bg}
			}
			\only<6->{
				Abstrakt symbolische Darstellung\\
				\includegraphics[width=.22\textwidth]{mediaDefense/vis/GlyphEventTypeAbstractBirthWhite-bg}
				\hspace*{.04\textwidth}%
				\includegraphics[width=.22\textwidth]{mediaDefense/vis/GlyphEventTypeAbstractDeathWhite-bg}
				\hspace*{.04\textwidth}%
				\includegraphics[width=.22\textwidth]{mediaDefense/vis/GlyphEventTypeAbstractMergeWhite-bg}
				\hspace*{.04\textwidth}%
				\includegraphics[width=.22\textwidth]{mediaDefense/vis/GlyphEventTypeAbstractSplitWhite-bg}
			}
		\end{column}
		\begin{column}{.5\textwidth}
			\begin{itemize}
				\item<1-> Rahmen ist geschlossen
				\item<1-> gutes Rand-Flächen-Verhältnis (Platz für Symbolik)
				\item<2-> Abheben von der Kugeldarstellung der Partikel
				\item<3-> Hinweis auf exakte Ereignisposition
				\item<5-> Ereignisart ohne Legende erkennbar
			\end{itemize}
		\end{column}
	\end{columns}
\end{frame}

%
% Verschiedene Darstellungsmethoden mit Vor- und Nachteilen für verschiedene Zoomstufen
% Textur: gut isoliert zu betrachten
% Farbwert: gut in der Makrosicht
%
\begin{frame}{Glyphdarstellung Strukturereignis: Zeitdarstellung}
	\visible<1->{
		Füllstand (Textur)\\
		\includegraphics[width=.14\textwidth]{mediaDefense/vis/GlyphenEventTimeFill000}
		\hspace*{.1\textwidth}%
		\includegraphics[width=.14\textwidth]{mediaDefense/vis/GlyphenEventTimeFill050}
		\hspace*{.1\textwidth}%
		\includegraphics[width=.14\textwidth]{mediaDefense/vis/GlyphenEventTimeFill100}
	}
	
	\visible<2->{
		Helligkeit\\
		\includegraphics[width=.14\textwidth]{mediaDefense/vis/GlyphenEventTimeBrightness000}
		\hspace*{.1\textwidth}%
		\includegraphics[width=.14\textwidth]{mediaDefense/vis/GlyphenEventTimeBrightness050}
		\hspace*{.1\textwidth}%
		\includegraphics[width=.14\textwidth]{mediaDefense/vis/GlyphenEventTimeBrightness100}
		\hspace*{.1\textwidth}%
		\includegraphics[width=.3\textwidth]{mediaDefense/vis/Farbskala-Helligkeit}
	}
	
	\visible<3->{
		Farbwert\\
		\includegraphics[width=.14\textwidth]{mediaDefense/vis/GlyphenEventTimeHue000}
		\hspace*{.1\textwidth}%
		\includegraphics[width=.14\textwidth]{mediaDefense/vis/GlyphenEventTimeHue050}
		\hspace*{.1\textwidth}%
		\includegraphics[width=.14\textwidth]{mediaDefense/vis/GlyphenEventTimeHue100}
		\hspace*{.1\textwidth}%
		\includegraphics[width=.3\textwidth]{mediaDefense/vis/Farbskala-Farbton}
	}
\end{frame}

\section{Ergebnisse}
%\frame{\tableofcontents[currentsection, hideothersubsections]}

%
% Ziel: Bewertung der Visualisierung
% Umfrage hat einfache Erklärungen und Fragestellungen -> Zielgruppe: Laien und Professionelle
% Abgeschlossen haben 50 TN: 8 Profis, 42 Laien
% Glyphdesign: intuitive Erkennung der Ereignisart, Zeitkodierung (Helligkeit: Erkennung ohne Farbskala) in Makrosicht, Positionierung
%
% Art: Erkennbarkeit hoch für beide Varianten (84+%)
% Zeit: intuitive Wertzuordnung bei Füllstand mit Abstand am Besten, Erkennung der Farbe in der Makrosicht mit Abstand am Besten bewertet (74%)
%
\subsection{Umfrage}
\begin{frame}{Zuordbarkeit der Kodierung}
	Es sind vier Glyphen zu sehen. Jeder Glyph steht für ein Strukturereignis. Bitte ordne die zufällig angeordneten Bilder nach ihrer Bedeutung:\\
	\includegraphics*[height=.7\textheight]{mediaDefense/eva/umfrage-ereignisart}
\end{frame}

%
% fachliche Interpretation Visualisierung
%
% Alle detektierten Ereignisse über den gesamten Zeitraum (ms2-40)
% Frage nach Herkunft der schnürenartigen Verläufe:
% Filamente, Tröpfchen, Fronten oder strukturunabhängig
% Experten hatte ihre Schwierigkeiten: 63% (5) lagen richtig, ((25% (2) sogar ganz falsch (strukturunabhängig)))
% Laien geraten (gleichverteilte Antworten)
%
\begin{wideframe}
	\includegraphics*[width=\textwidth]{mediaDefense/eva/EventFlow-Hue-5r-10-ms2-40-bd2-cropped-zeilen}
%TODO Fragestellung, Antwortmöglichkeiten: Filamente, Tröpfchen, Fronten, strukturunabhängig
\end{wideframe}

\subsection{Zusammenfassung, Bewertung und Ausblick}
\begin{wideframe}
	\includegraphics*[width=\textwidth]{mediaDefense/VisMD-Ablaufschema-komplett}
\end{wideframe}

% viele stattfindende Ereignisse können erfolgreich detektiert werden
% Unterscheidung von Fehldetektionen und tatsächlichen Ereignissen schwierig: nur in der direkten Überlagerung mit der Partikelansicht möglich
\begin{wideframe}
	{\scriptsize $t_S=101$}
	\hspace{.41\textwidth}
	{\scriptsize $t_S=102$}
	
	\includegraphics*[width=.5\textwidth]{mediaDefense/eva/spliterkennung-ms2-40-f101-small}
	\hspace{.01\textwidth}
	\includegraphics*[width=.5\textwidth]{mediaDefense/eva/spliterkennung-ms2-40-f102-small}
%TODO besseres Beispiel finden oder sagen, warum Split schon bei Zeitschritt 101 auftaucht
	\begin{itemize}[<+->]
		\item stattfindende Ereignisse können detektiert und visualisiert werden
		\item Unterscheidung von Fehldetektionen und tatsächlichen Ereignissen schwierig
	\end{itemize}
\end{wideframe}

%
% Ausblick
% weiteres:
% Filterung anhand Clustergröße
% größere Mindestclustergrößen (weniger Fehldetektionen)
% Nebeneinanderstellung mit Beziehung zwischen den Elementen (explizite Verschlüsselung)
%
\begin{frame}{Ausblick}
	\begin{itemize}
		\item Betrachtung über mehrere Zeitschritte $\rightarrow$ Ereignisse können miteinander in Beziehung gesetzt werden
		\item Clusternachbarschaftsgraph ergänzend zur Heuristik
		\item Nutzung von \wichtig{Konturbäumen} anstatt des \wichtig{CFD}
	\end{itemize}
\end{frame}

\begin{frame}{Fragen}
	Fragen?
\end{frame}

\end{document}

