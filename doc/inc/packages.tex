\RequirePackage[ngerman=ngerman-x-latest]{hyphsubst}
\documentclass[english,ngerman]{tudscrreprt}
\usepackage{babel}
\usepackage{selinput}
\SelectInputMappings{adieresis={ä},germandbls={ß}}
\usepackage[T1]{fontenc}
\usepackage{fixltx2e}
\usepackage{scrhack}
\usepackage{tudscrsupervisor}
\usepackage{xcolor}
\usepackage{gensymb}

%%%
% Glossar
%
\AfterPackage*{hyperref}{%
	\usepackage[%
		automake,% Alphabetische Sortierung, da xindy aktiv, kein makeindex benötigt
		% mit Tex Live einfach verwendbar
		xindy={language=german-din}, % Alphabetische Sortierung nach UTF-8 und Duden oder DIN.
		acronym,% Abkürzungen
		symbols,% Formelzeichen
		%nomain,% kein Glossar
		translate=babel,% Überschriften der Glossare in der Dokumentsprache gesetzt
		nogroupskip,% automatischer Abstand zwischen den Einträgen zur Gruppierung innerhalb eines Glossars entfernen
		toc,% fügt die Verzeichnisse dem Inhaltsverzeichnis hinzu
		section=chapter,% bestimmt die Gliederungsebene der Überschrift
	]{glossaries}
	\makeglossaries
	
\newglossarystyle{acrotabu}{%
	\renewenvironment{theglossary}{%
		\begin{tabu}spread 0pt{@{}lX<{\strut}l@{}}%
	}{%
		\end{tabu}\par\bigskip%
	}%
	\renewcommand*{\glossaryheader}{}%
	\renewcommand*{\glsgroupheading}[1]{}%
	\renewcommand*{\glsgroupskip}{}%
	\renewcommand*{\glossentry}[2]{%
		\glsentryitem{##1}% Entry number if required
		\glstarget{##1}{\sffamily\bfseries\glossentryname{##1}} &
		\glsentrydesc{##1} &
		##2\tabularnewline
	}
}

\newcommand*{\newsymbol}[5][]{%
	\newglossaryentry{#2}{%
		type=symbols,%
		description={},%
		name={#3},%
		symbol={\ensuremath{#4}},%
		user1={\ensuremath{\mathrm{#5}}},%
		sort={#2},%
		#1%
	}%
}

\defglsentryfmt[symbols]{%
	\ifmmode%
	\glssymbol{\glslabel}%
	\else%
	\glsgenentryfmt~\glsentrysymbol{\glslabel}%
	\fi%
}
\newglossarystyle{symblongtabu}{%
	\renewenvironment{theglossary}{%
		\begin{longtabu}spread 0pt[l]{ccX<{\strut}l}%
		}{%
		\end{longtabu}%
	}%
	\renewcommand*{\glossaryheader}{%
		\toprule
		\bfseries Symbol & \bfseries Einheit &
		\bfseries Name & \bfseries Seite(n)
		\tabularnewline\midrule\endhead%
		\bottomrule\endfoot%
	}%
	\renewcommand*{\glsgroupheading}[1]{}%
	\renewcommand*{\glsgroupskip}{}%
	\renewcommand*{\glossentry}[2]{%
		\glsentryitem{##1}% Entry number if required
		\glstarget{##1}{\glossentrysymbol{##1}} &
		\glsentryuseri{##1} &
		\glossentryname{##1} &
		##2\tabularnewline%
	}%
}
}% Ende von AfterPackage*

%%%
% Zitate
%
\usepackage{csquotes}
\usepackage[backend=biber,style=alphabetic]{biblatex}
\addbibresource{bib/allgemein.bib}
\addbibresource{bib/countourTree.bib}
\addbibresource{bib/glyphen.bib}
\addbibresource{bib/grottel.bib}
\addbibresource{bib/polyeder.bib}
\addbibresource{bib/skeletonExtraction.bib}
\addbibresource{bib/zusammenhangskomponenten.bib}

%%%
% Caption
%
\usepackage{caption}
\captionsetup{font=sf,labelfont=bf,labelsep=space}
\usepackage{floatrow}
\floatsetup{font=sf}
\floatsetup[table]{style=plaintop}
\captionsetup{singlelinecheck=off,format=hang,justification=raggedright}
\DeclareCaptionSubType[alph]{figure}
\DeclareCaptionSubType[alph]{table}
\captionsetup[subfloat]{labelformat=brace,list=off}

%%%
% Tabellen
%
\usepackage{booktabs} % Linien für Tabellen.
\usepackage{array} % Definitionen für Spalten.
\usepackage{tabularx} % Tabellen mit gleicher Spaltenbreite.
\usepackage{tabulary}
\usepackage{tabu}


%%%
% Index
%
\usepackage{imakeidx}
\indexsetup{%
	level=\chapter*,
	noclearpage, firstpagestyle=headings, headers={\indexname}{\indexname},
	othercode={\renewcommand*\subitem{\@idxitem\hspace*{15\p@}}}
}\makeindex

%%%
% Bilder
%
\usepackage{graphicx}
\graphicspath{ {./media/} }

%%%
% Quellcode
%
\usepackage{listings}
% UTF8
\lstset{%
	inputencoding = utf8,
	extendedchars = true, % lets you use non-ASCII characters; for 8-bits encodings only, does not work with UTF-8
	literate=%
	{ä}{{\"a}}1 {ö}{{\"o}}1 {ü}{{\"u}}1
	{Ä}{{\"A}}1 {Ö}{{\"O}}1 {Ü}{{\"U}}1
	{~}{{\textasciitilde}}1 {ß}{{\ss}}1
}
\definecolor{dkgreen}{rgb}{0,0.6,0}
\definecolor{mygray}{rgb}{0.5,0.5,0.5}
\definecolor{mymauve}{rgb}{0.58,0,0.82}
\lstset{ %
	backgroundcolor=\color{white},   % choose the background color; you must add \usepackage{color} or \usepackage{xcolor}
	basicstyle=\footnotesize,        % the size of the fonts that are used for the code
	breakatwhitespace=false,         % sets if automatic breaks should only happen at whitespace
	breaklines=true,                 % sets automatic line breaking
	captionpos=b,                    % sets the caption-position to bottom
	commentstyle=\color{dkgreen},    % comment style
	deletekeywords={...},            % if you want to delete keywords from the given language
	escapeinside={\%*}{*)},          % if you want to add LaTeX within your code
	frame=single,                    % adds a frame around the code
	keepspaces=true,                 % keeps spaces in text, useful for keeping indentation of code (possibly needs columns=flexible)
	keywordstyle=\color{blue},       % keyword style
	language=Octave,                 % the language of the code
	morekeywords={*,...},            % if you want to add more keywords to the set
	numbers=left,                    % where to put the line-numbers; possible values are (none, left, right)
	numbersep=5pt,                   % how far the line-numbers are from the code
	numberstyle=\tiny\color{mygray}, % the style that is used for the line-numbers
	rulecolor=\color{black},         % if not set, the frame-color may be changed on line-breaks within not-black text (e.g. comments (green here))
	showspaces=false,                % show spaces everywhere adding particular underscores; it overrides 'showstringspaces'
	showstringspaces=false,          % underline spaces within strings only
	showtabs=false,                  % show tabs within strings adding particular underscores
	stepnumber=2,                    % the step between two line-numbers. If it's 1, each line will be numbered
	stringstyle=\color{mymauve},     % string literal style
	tabsize=2,                       % sets default tabsize to 2 spaces
	title=\lstname                   % show the filename of files included with \lstinputlisting; also try caption instead of title
}

%%%
% Links
%
\usepackage{hyperref} % Möglichst am Ende stehen, da es viele Befehle neu definiert.
\hypersetup{
	colorlinks   = true, %Colours links instead of ugly boxes
	urlcolor     = HKS41!70, %Colour for external hyperlinks
	linkcolor    = HKS44!70, %Colour of internal links
	citecolor    = HKS33!80 %Colour of citations
}

\usepackage{quoting}

\usepackage[babel]{microtype}

\usepackage{xfrac}

%%%
% Aufzählungen
%
\usepackage{enumitem}
\setlist[itemize]{noitemsep}
\setlist[description]{noitemsep}

\usepackage{isodate}

\usepackage{ellipsis}
\let\ellipsispunctuation\relax

\usepackage{amssymb} % Muss nach hyperref kommen, sonst Fehler "\newsymbol already defined". Alternative: http://tex.stackexchange.com/questions/73684/are-there-any-situtations-where-it-is-better-to-load-a-package-after-other-code

\usepackage{tikz}
\usetikzlibrary{arrows,positioning,scopes,trees}