%%%
% Symbole, \glssymbol benutzen!
%

\glossarysymbol%First argument is newline, yeah.
	{R3}%
	{\mathbb{R}^3}%
	{Dreidimensionaler Raum.}
	{\emptyset}

\glossarysymbol%
	{reaModus}%
	{r_\text{eaModus}}%
	{Der Einstellungsparameter zum Filtern der anzuzeigenden Ereignisarten des \MMStrukturs{StaticRenderer Moduls}{StaticRenderer Modul}.}%
	{\emptyset}

\glossarysymbol%
	{rvisZeit}%
	{r_\text{visZeit}}%
	{Der Einstellungsparameter für die drei Varianten der visuellen Zeitkodierung von Ereignissen des \MMStrukturs{StaticRenderer Moduls}{StaticRenderer Modul}.}%
	{\emptyset}

\glossarysymbol%
	{rvisPos}%
	{r_\text{visPos}}%
	{Der Einstellungsparameter zur Änderung der visuellen Positionskodierung von Ereignissen des \MMStrukturs{StaticRenderer Moduls}{StaticRenderer Modul}.}%
	{\emptyset}

\glossarysymbol%
	{rvisArt}%
	{r_\text{visArt}}%
	{Der Einstellungsparameter zur Änderung der visuellen Artkodierung von Ereignissen des \MMStrukturs{StaticRenderer Moduls}{StaticRenderer Modul}.}%
	{\emptyset}

\glossarysymbol%
	{rglyph}%
	{r_\text{Glyph}}%
	{Der Einstellungsparameter für die Glyphgröße des \MMStrukturs{StaticRenderer Moduls}{StaticRenderer Modul}.}%
	{\emptyset}

\glossarysymbol%
	{rzeit}%
	{r_\text{zeit}}%
	{Der Einstellungsparameter für die anzuzeigende Zeitspanne des \MMStrukturs{StaticRenderer Moduls}{StaticRenderer Modul}.}%
	{\emptyset}

\glossarysymbol%
	{rzModus}%
	{r_\text{zModus}}%
	{Der Einstellungsparameter für verschiedene Zeitanzeigemodi des \MMStrukturs{StaticRenderer Moduls}{StaticRenderer Modul}.}%
	{\emptyset}

\glossarysymbol%
	{cradMod}%
	{c_\text{radMod}}%
	{Der Einstellungsparameter \CFDterm{Partikelradiusmultiplikator}. Er bestimmt die Reichweite der Nachbarschaftssuche der Partikel.}%
	{\emptyset}

\glossarysymbol%
	{cminCG}%
	{c_\text{minCG}}%
	{Einstellungsparameter der Clusterreduktion für die Mindestclustergröße.}%
	{Partikelanzahl (\emptyset)}

\glossarysymbol%
	{ebd}%
	{e_\text{bd}}%
	{Einstellungsparameter der \SECterm{Ereignisheuristik} für die Gesamtanteile an gemeinsamen Partikeln.}%
	{\%}

\glossarysymbol%
	{emsA}%
	{e_\text{msA}}%
	{Einstellungsparameter der \SECterm{Ereignisheuristik} für die Anteile an gemeinsamen Partikeln der \SECterms{Großen Partner}{Großer Partner}.}%
	{\%}
	
\glossarysymbol%
	{emsP}%
	{e_\text{msP}}%
	{Einstellungsparameter der \SECterm{Ereignisheuristik} für die Anzahl der \SECterm{Partnercluster}}%
	{Clusteranzahl (\emptyset)}
	
%%%
% Akronyme
%

\newacronym[
longplural={CSV}
description={Comma Separated Values.}
]{csv}{CSV}{Comma Separated Values}

\newacronym[
	longplural={Relationale Datenbankmanagementsysteme}
	description={Relationales Datenbankmanagementsystem}
]{rdbms}{RDBMS}{Relationalen Datenbankmanagementsystems}

\newacronym[
	longplural={Hue-Saturation-Brightness}
	description={Ein Farbschema mit drei Dimensionen für Farbwert (0°-360°), Sättigung und Helligkeit.}
]{hsb}{HSB}{Hue-Saturation-Brightness}

\newacronym[
longplural={Hue-Saturation-Value}
description={Ein Farbschema mit drei Dimensionen für Farbwert (0°-360°), Sättigung und Helligkeit.}
]{hsv}{HSV}{Hue-Saturation-Value}

\newacronym[
longplural={Rot-Grün-Blau}
description={Ein Farbschema mit drei Dimensionen für die Farben rot, grün und blau.}
]{rgb}{RGB}{Rot-Grün-Blau}

\newacronym[
longplural={Pixel}
description={Eine Einheit.}
]{px}{px}{Pixel}

\newacronym[
longplural={OpenGL}
description={Eine Schnittstelle für die graphische Ausgabe.}
]{ogl}{OGL}{OpenGL}

\newacronym[
longplural={OpenGL Mathematics}
description={Eine headeronly Bibliothek.}
]{glm}{glm}{OpenGL Mathematics}

\newacronym[
longplural={Smoothed-particle hydrodynamics}
description={Numerische Methode, um die Hydrodynamischen Gleichungen zu lösen.}
]{sph}{SPH}{Smoothed-particle hydrodynamics}

\newacronym[
longplural={Portable Network Graphics}
description={Ein verlustfreies Bildformat.}
]{png}{PNG}{Portable Network Graphics}

\newacronym[
longplural={Millisekunden}
description={Zeitangabe in Millisekunden.}
]{ms}{ms}{Millisekunden}

\newacronym[
longplural={Kibibyte}
description={Eine Einheit.}
]{kiB}{kiB}{Kibibyte}

\newacronym[
longplural={Gibibyte}
description={Eine Einheit.}
]{GiB}{GiB}{Gibibyte}

\newacronym[
longplural={MIT}
description={Eine OpenSource Lizenz ursprünglich entstanden am Massachusetts Institute of Technology.}
]{MIT}{MIT}{MIT Lizenz}

\newacronym[
longplural={XML}
description={Extensible Markup Language.}
]{xml}{XML}{Extensible Markup Language}

\newacronym[
longplural={Fast Library for Approximate Nearest Neighbors}
description={Fast Library for Approximate Nearest Neighbors nutzt kD-Bäume.}
]{FLANN}{FLANN}{Fast Library for Approximate Nearest Neighbors}

\newacronym[
longplural={Parallel Patterns Libraries}
description={Bibliothek von Microsoft zur Parallelisierung von C++ Code.}
]{PPL}{PPL}{Parallel Patterns Library}

%%%
% MegaMol
%

\newacronym[
longplural={Approximate Nearest Neighbor Bibliotheken}
description={Approximate Nearest Neighbor Bibliothek nutzt einen kD-Baum.}
]{ANN}{ANN}{Approximate Nearest Neighbor Bibliothek}

\newacronym[
longplural={Cluster Fast Depth Algorithmus}
description={Der Algorithmus für die Clusterberechnung.}
]{cfd}{CFD}{Cluster Fast Depth Algorithmus}

\newacronym[
longplural={Determine Structure Events}
description={Methodik zur Erkennung von Strukturereignissen.}
]{dse}{DSE}{Determine Structure Events}

\newacronym[
longplural={Multi Particle Data Call}
description={Der Call für die Partikellisten.}
]{mpdc}{MPDC}{Multi Particle Data Call}

\newacronym[
longplural={MegaMol Multi Particle List Dump}
description={Das Dateiformat für die Partikellisten.}
]{mmpld}{MMPLD}{MegaMol Multi Particle List Dump}

\newacronym[
longplural={MegaMol Structure Events}
description={Das Dateiformat für die Structure Events.}
]{mmse}{MMSE}{MegaMol Structure Event}

\newacronym[
longplural={Structure Events Calculation Modul}
description={Berechnungsmodul.}
]{SECalc}{SECalc}{Structure Events Calculation Modul}

\newacronym[
longplural={Structure Events Cluster Compare}
description={Vergleich von Clustern über zwei Frames.}
]{secc}{SECC}{Structure Events Cluster Compare}

\newacronym[
longplural={Structure Events Data Call}
description={Der MegaMol Call, der zum Übertragen von Structure Events Daten genutzt wird.}
]{sedc}{SEDC}{Structure Events Data Call}