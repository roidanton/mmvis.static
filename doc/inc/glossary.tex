\newglossaryentry{platonischer Körper}{
	name = {platonischer Körper},
	description = {fünf},
	plural = {platonische Körper},
	sort = {Körper}
}

\newglossaryentry{3dRaum} {
	name = {\ensuremath{\mathbb{R}^3}},
	description = {Dreidimensionaler Raum},
	sort = {R}
}

\newacronym[
	longplural={Relationale Datenbankmanagementsysteme}
	description={Relationales Datenbankmanagementsystem}
]{rdbms}{RDBMS}{Relationalen Datenbankmanagementsystems}

\newacronym[
	longplural={Hue-Saturation-Brightness}
	description={Ein Farbschema mit drei Dimensionen für Farbwert (0°-360°), Sättigung und Helligkeit.}
]{hsb}{HSB}{Hue-Saturation-Brightness}

\newacronym[
longplural={Hue-Saturation-Value}
description={Ein Farbschema mit drei Dimensionen für Farbwert (0°-360°), Sättigung und Helligkeit.}
]{hsv}{HSV}{Hue-Saturation-Value}

\newacronym[
longplural={Rot-Grün-Blau}
description={Ein Farbschema mit drei Dimensionen für die Farben rot, grün und blau.}
]{rgb}{RGB}{Rot-Grün-Blau}

\newacronym[
longplural={Pixel}
description={Eine Einheit.}
]{px}{px}{Pixel}

\newacronym[
longplural={Millisekunden}
description={Zeitangabe in Millisekunden.}
]{ms}{ms}{Millisekunden}

\newacronym[
longplural={Kibibyte}
description={Eine Einheit.}
]{kiB}{kiB}{Kibibyte}

\newacronym[
longplural={Gibibyte}
description={Eine Einheit.}
]{GiB}{GiB}{Gibibyte}

\newacronym[
longplural={MIT}
description={Eine OpenSource Lizenz ursprünglich entstanden am Massachusetts Institute of Technology.}
]{MIT}{MIT}{MIT Lizenz}

\newacronym[
longplural={XML}
description={Extensible Markup Language.}
]{XML}{XML}{Extensible Markup Language}

\newacronym[
longplural={Fast Library for Approximate Nearest Neighbors}
description={Fast Library for Approximate Nearest Neighbors nutzt kD-Bäume.}
]{FLANN}{FLANN}{Fast Library for Approximate Nearest Neighbors}

\newacronym[
longplural={Parallel Patterns Libraries}
description={Bibliothek von Microsoft zur Parallelisierung von C++ Code.}
]{PPL}{PPL}{Parallel Patterns Library}

%%%
% MegaMol
%

\newacronym[
longplural={Approximate Nearest Neighbor Bibliotheken}
description={Approximate Nearest Neighbor Bibliothek nutzt einen kD-Baum.}
]{ANN}{ANN}{Approximate Nearest Neighbor Bibliothek}

\newacronym[
longplural={Cluster Fast Depth Algorithmus}
description={Der Algorithmus für die Clusterberechnung.}
]{cfd}{CFD}{Cluster Fast Depth Algorithmus}

\newacronym[
longplural={Determine Structure Events}
description={Methodik zur Erkennung von Strukturereignissen.}
]{dse}{DSE}{Determine Structure Events}

\newacronym[
longplural={Multi Particle Data Call}
description={Der Call für die Partikellisten.}
]{mpdc}{MPDC}{Multi Particle Data Call}

\newacronym[
longplural={MegaMol Multi Particle List Dump}
description={Das Dateiformat für die Partikellisten.}
]{mmpld}{MMPLD}{MegaMol Multi Particle List Dump}

\newacronym[
longplural={MegaMol Structure Events}
description={Das Dateiformat für die Structure Events.}
]{mmse}{MMSE}{MegaMol Structure Event}

\newacronym[
longplural={Structure Events Calculation Modul}
description={Berechnungsmodul.}
]{SECalc}{SECalc}{Structure Events Calculation Modul}

\newacronym[
longplural={Structure Events Cluster Compare}
description={Vergleich von Clustern über zwei Frames.}
]{secc}{SECC}{Structure Events Cluster Compare}

\newacronym[
longplural={Structure Events Data Call}
description={Der MegaMol Call, der zum Übertragen von Structure Events Daten genutzt wird.}
]{sedc}{SEDC}{Structure Events Data Call}