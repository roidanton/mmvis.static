%%%
% Index
%
\newcommand{\CFD}{\gls{cfd}\index{CFD-Algorithmus} }
\newcommand{\MMSE}{\gls{mmse}\index{MMSE-Datei} }
\newcommand{\SECC}{\gls{secc}\index{SECC-Algorithmus} }
\newcommand{\SEDC}{\gls{sedc}\index{StructureEventsDataCall} }
\newcommand{\SECalc}{\gls{SECalc}\index{StructureEventsCalculation Modul} }
\newcommand{\ANN}{\gls{ANN}-Bibliothek\index{ANN-Bibliothek} }
\newcommand{\FLANN}{\gls{FLANN}-Bibliothek\index{FLANN-Bibliothek} }
\newcommand{\PPL}{\gls{PPL}-Bibliothek\index{Parallel Patterns Library} }
\newcommand{\OGL}{\gls{ogl}\index{OpenGL} }


%Allgemein
\newcommand{\tool}[1]{\textit{#1}\index{#1}} %und libs. Wg vieler Abkürzungen evtl ins Glossar! (z.B. ANN, OpenMP, PPL)
\newcommand{\tools}[2]{\textit{#1}\index{#2}}
\newcommand{\MMStruktur}[1]{\textit{#1}\index{#1}} % Programmaufbau, evtl in Glossar umwandeln!
\newcommand{\MMStrukturs}[2]{\textit{#1}\index{#2}} % Programmaufbau.
\newcommand{\vis}[1]{\textit{#1}\index{#1}} %Visualisierung
\newcommand{\viss}[2]{\textit{#1}\index{#2}} %Visualisierung

% Berechnung
\newcommand{\ereignis}[1]{\textit{#1}\index{#1}}
\newcommand{\ereigniss}[2]{\textit{#1}\index{#2}}
\newcommand{\contour}[1]{\textit{#1}\index{#1}}
\newcommand{\contours}[2]{\textit{#1}\index{#2}}
\newcommand{\SECterm}[1]{\textit{#1}\index{#1}}
\newcommand{\SECterms}[2]{\textit{#1}\index{#2}}
\newcommand{\CFDterm}[1]{\textit{#1}\index{#1}}
\newcommand{\CFDterms}[2]{\textit{#1}\index{#2}}

%%%
% Auszeichnungen
%
\newcommand{\wichtig}[1]{\textit{#1}}
\newcommand{\mrkg}[1]{\textcolor{orange}{#1}} % Markierung, etwa in Gleichungen
\newcommand{\mrkgb}[1]{\textcolor{magenta}{#1}} % Markierung, etwa in Gleichungen

%%%
% Tabelle
%
\newcolumntype{F}{>{\hspace{0pt}}X}
\newcolumntype{L}{>{\raggedright}F}
\newcolumntype{C}{>{\centering}F}
\newcolumntype{D}{>{\centering}F}
\newcolumntype{R}{>{\raggedleft}F}

%%%
% Symbole
%
\newcommand{\kreuz}{ $\times$ }
\newcommand{\esfolgt}{ $\to$ }
\newcommand{\eqtxt}[1]{\stackrel{\text{#1}}{=}} %Gleichheitszeichen mit Text drüber
\newcommand{\intxt}[1]{\stackrel{\text{#1}}{\in}} %Element von mit Text drüber
\newcommand{\abs}[1]{\lvert#1\rvert}
\newcommand{\norm}[1]{\lVert#1\rVert}
\newcommand{\qed}{$_\square$} %Was zu beweisen war; quod erat demonstrandum

%%%
% TikZ
%
\tikzset {
	value/.style={rectangle, draw},
	vocab/.style={rectangle, draw, rounded corners=.8ex},
	verticalChild/.style={grow=down, xshift=.5em, anchor=west, font = \small,
					edge from parent path={(\tikzparentnode.south) |- (\tikzchildnode.west)}},
	first/.style={level distance=6ex},
	second/.style={level distance=12ex},
	third/.style={level distance=18ex},
	fourth/.style={level distance=24ex},
	level 1/.style={sibling distance=3cm}
}